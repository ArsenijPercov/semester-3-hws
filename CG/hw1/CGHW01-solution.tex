\documentclass[a4paper]{article}
\usepackage[pdftex]{hyperref}
\usepackage[latin1]{inputenc}
\usepackage[english]{babel}
\usepackage{a4wide}
\usepackage{amsmath}
\usepackage{amssymb}
\usepackage{algorithmic}
\usepackage{algorithm}
\usepackage{ifthen}
\usepackage{listings}
% move the asterisk at the right position
\lstset{basicstyle=\ttfamily,tabsize=4,literate={*}{${}^*{}$}1}
%\lstset{language=C,basicstyle=\ttfamily}
\usepackage{moreverb}
\usepackage{palatino}
\usepackage{multicol}
\usepackage{tabularx}
\usepackage{comment}
\usepackage{verbatim}
\usepackage{color}
\usepackage{enumitem}
\usepackage{tikz}
\usetikzlibrary{arrows,shapes.gates.logic.US,shapes.gates.logic.IEC,calc}

\usepackage[left=3cm, right=3cm, top=4cm, bottom=4cm]{geometry}

\usepackage{graphicx}

%% pdflatex?
\newif\ifpdf
\ifx\pdfoutput\undefined
\pdffalse % we are not running PDFLaTeX
\else
\pdfoutput=1 % we are running PDFLaTeX
\pdftrue
\fi
\ifpdf
\usepackage[pdftex]{graphicx}
\else
\usepackage{graphicx}
\fi
\ifpdf
\DeclareGraphicsExtensions{.pdf, .jpg}
\else
\DeclareGraphicsExtensions{.eps, .jpg}
\fi

\parindent=0cm
\parskip=0cm

\setlength{\columnseprule}{0.4pt}
\addtolength{\columnsep}{2pt}

\addtolength{\textheight}{5.5cm}
\addtolength{\topmargin}{-26mm}
\pagestyle{empty}

%%
%% Sheet setup
%% 
\newcommand{\coursename}{Computer Graphics}
\newcommand{\courseno}{CO19-320322}
 
\newcommand{\sheettitle}{Problem Sheet}
\newcommand{\mytitle}{}
\newcommand{\mytoday}{March 22, 2019}

% Current Assignment number
\newcounter{assignmentno}
\setcounter{assignmentno}{1}

% Current Problem number, should always start at 1
\newcounter{problemno}
\setcounter{problemno}{1}

%%
%% problem and bonus environment
%%
\newcounter{probcalc}
\newcommand{\problem}[2]{
  \pagebreak[2]
  \setcounter{probcalc}{#2}
  ~\\
  {\large \textbf{Problem \arabic{assignmentno}.\arabic{problemno}} \hspace{0.2cm}\textit{#1}} \refstepcounter{problemno}\vspace{2pt}\\}

\newcommand{\bonus}[2]{
  \pagebreak[2]
  \setcounter{probcalc}{#2}
  ~\\
  {\large \textbf{Bonus Problem \textcolor{blue}{\arabic{assignmentno}}.\textcolor{blue}{\arabic{problemno}}} \hspace{0.2cm}\textit{#1}} \refstepcounter{problemno}\vspace{2pt}\\}

%% some counters  
\newcommand{\assignment}{\arabic{assignmentno}}

%% solution  
\newcommand{\solution}{\pagebreak[2]{\bf Solution:}\\}

%% Hyperref Setup
\hypersetup{pdftitle={Homework \assignment},
  pdfsubject={\coursename},
  pdfauthor={},
  pdfcreator={},
  pdfkeywords={Computer Graphics},
  %  pdfpagemode={FullScreen},
  %colorlinks=true,
  %bookmarks=true,
  %hyperindex=true,
  bookmarksopen=false,
  bookmarksnumbered=true,
  breaklinks=true,
  %urlcolor=darkblue
  urlbordercolor={0 0 0.7}
}

\begin{document}
\coursename \hfill Course: \courseno\\
Jacobs University Bremen \hfill \mytoday\\
Dushan Terzikj\hfill
\vspace*{0.3cm}\\
\begin{center}
{\Large \sheettitle{} \assignment\\}
\end{center}

\problem{}{0}
\solution
\begin{enumerate}
    \item True
    \item True
    \item False
    \item True
    \item False
\end{enumerate}

\problem{}{0}
\solution
\begin{enumerate}[label=(\alph*)]
    \item First we need to derive the scaling matrix.
    \[S=\begin{bmatrix}
    3&0&0\\
    0&-1&0\\
    0&0&1\\
    \end{bmatrix}\]
    Then we need to derive the rotation matrix around the z-axis:
    \[R=\begin{bmatrix}
    cos\theta&-sin\theta&0\\
    sin\theta&cos\theta&0\\
    0&0&1\\
    \end{bmatrix}\]
    Since $\theta=\frac{\pi}{2}$, we have:
    \[R=\begin{bmatrix}
    0&-1&0\\
    1&0&0\\
    0&0&1\\
    \end{bmatrix}\]
    Now we need to derive the translation matrix $T$. Since translation is affine, we need to derive it in homogeneous coordinates. The next follows:
    \[T=\begin{bmatrix}
    1&0&0&b_x\\
    0&1&0&0\\
    0&0&1&0\\
    0&0&0&1\\
    \end{bmatrix}
    =\begin{bmatrix}
    1&0&0&2\\
    0&1&0&0\\
    0&0&1&0\\
    0&0&0&1\\
    \end{bmatrix}\]
    Once we make the rotation matrix $R$ and the scaling matrix $S$ have homogeneous coordinates, we can combine all three matrices to achieve the full transformation matrix $T_t$:\\
    \[T_t=TRS=
    \begin{bmatrix}
    1&0&0&2\\
    0&1&0&0\\
    0&0&1&0\\
    0&0&0&1\\
    \end{bmatrix}
    \begin{bmatrix}
    0&-1&0&0\\
    1&0&0&0\\
    0&0&1&0\\
    0&0&0&1\\
    \end{bmatrix}
    \begin{bmatrix}
    3&0&0&0\\
    0&-1&0&0\\
    0&0&1&0\\
    0&0&0&1\\
    \end{bmatrix}
    =\begin{bmatrix}
    0&-1&0&2\\
    -3&0&0&0\\
    0&0&1&0\\
    0&0&0&1\\
    \end{bmatrix}
    \]
    Once we make our vectors $p_1, p_2, p_3$ homogeneous and put them in a matrix, where its column is a vector $p_i, i=1,2,3$, we can multiply it with $T_t$ which will yield 3 vectors which represent the transformed triangle points:
    \[\begin{bmatrix}
    0&-1&0&2\\
    -3&0&0&0\\
    0&0&1&0\\
    0&0&0&1\\
    \end{bmatrix}
    \begin{bmatrix}
    3&2&1\\
    0&0&1\\
    2&2&2\\
    1&1&1\\
    \end{bmatrix}
    =\begin{bmatrix}
    2&2&1\\
    -9&-6&-3\\
    2&2&2\\
    1&1&1\\
    \end{bmatrix}
    \]
    So the corresponding translated vectors $p_1^{'}, p_2^{'}, p_3^{'}$ are $[2, -9, 2]^T$, $[2, -6, 2]^T$, $[1, -3, 2]^T$ respectively.
    \item Projection matrix P has the following form:
    \[P=\begin{bmatrix}
    h&0&0&0\\
    0&h&0&0\\
    0&0&a&b\\
    0&0&1&0\\
    \end{bmatrix}\]
    In our case $a=b=1$. Additionally, $h=1$, because the distance from the camera (at point $[0, 0, 1]^T$) and the screen (at point $[0, 0, 0]^T$) is 1.
    \[P=\begin{bmatrix}
    1&0&0&0\\
    0&1&0&0\\
    0&0&1&1\\
    0&0&1&0\\
    \end{bmatrix}\]
    Now we can apply the projection matrix to the triangle points to project the triangle in 2-D.
    \[\begin{bmatrix}
    1&0&0&0\\
    0&1&0&0\\
    0&0&1&1\\
    0&0&1&0\\
    \end{bmatrix}
    \begin{bmatrix}
    2&2&1\\
    -9&-6&-3\\
    2&2&2\\
    1&1&1\\
    \end{bmatrix}
    =\begin{bmatrix}
    2&2&1\\
    -9&-6&-3\\
    3&3&3\\
    2&2&2\\
    \end{bmatrix}
    \]
    To get the actual coordinates, we have to normalize the 4th dimension. In this case the all three vectors can be multiplied by $\frac{1}{2}$ in order to be normalized.
    \[\Rightarrow\begin{bmatrix}
    1&1&\frac{1}{2}\\
    -\frac{9}{2}&-3&-\frac{3}{2}\\
    \frac{3}{2}&\frac{3}{2}&\frac{3}{2}\\
    1&1&1\\
    \end{bmatrix}\]
    The coordinates on the 2-D plane are $[1, -\frac{9}{2}]^T$, $[1, -3]^T$ and $[\frac{1}{2}, -\frac{3}{2}]^3$. The points go out of bounds and this means that they are not visible on the screen.
\end{enumerate}

\problem{}{0}
\solution
Let the $C$ matrix contain the vectors (column-wise) which represent the points of the cube. Then we have the following matrix (homogeneous coordinate system):
\[C=\begin{bmatrix}
1&1&-1&-1&1&1&-1&-1\\
1&-1&-1&1&1&-1&-1&1\\
1&1&1&-1&-1&-1&-1&1\\
1&1&1&1&1&1&1&1\\
\end{bmatrix}
\]
The rotation matrix $H_{ry}$ and the translation matrix $T_x$ are the following (homogeneous coordinate system):
\[H_{ry}=\begin{bmatrix}
cos\theta&0&sin\theta&0\\
0&1&0&0\\
-sin\theta&0&cos\theta&0\\
0&0&0&1\\
\end{bmatrix}
=\begin{bmatrix}
\frac{\sqrt{3}}{2}&0&\frac{1}{2}&0\\
0&1&0&0\\
-\frac{1}{2}&0&\frac{\sqrt{3}}{2}&0\\
0&0&0&1\\
\end{bmatrix}\]
\[T_x=\begin{bmatrix}
1&0&0&\frac{1}{2}\\
0&1&0&0\\
0&0&1&0\\
0&0&0&1\\
\end{bmatrix}\]
The complete transformation matrix is the following:
\[T=T_xH_{ry}=\begin{bmatrix}
1&0&0&\frac{1}{2}\\
0&1&0&0\\
0&0&1&0\\
0&0&0&1\\
\end{bmatrix}
\begin{bmatrix}
\frac{\sqrt{3}}{2}&0&\frac{1}{2}&0\\
0&1&0&0\\
-\frac{1}{2}&0&\frac{\sqrt{3}}{2}&0\\
0&0&0&1\\
\end{bmatrix}
=\begin{bmatrix}
\frac{\sqrt{3}}{2}&0&\frac{1}{2}&\frac{1}{2}\\
0&1&0&0\\
-\frac{1}{2}&0&\frac{\sqrt{3}}{2}&0\\
0&0&0&1\\
\end{bmatrix}
\]
\[
TC=\begin{bmatrix}
\frac{\sqrt{3}}{2}&0&\frac{1}{2}&\frac{1}{2}\\
0&1&0&0\\
-\frac{1}{2}&0&\frac{\sqrt{3}}{2}&0\\
0&0&0&1\\
\end{bmatrix}
\begin{bmatrix}
1&1&-1&-1&1&1&-1&-1\\
1&-1&-1&1&1&-1&-1&1\\
1&1&1&-1&-1&-1&-1&1\\
1&1&1&1&1&1&1&1\\
\end{bmatrix}
\]
\[
=\begin{bmatrix}
1.87&1.87&0.13&-0.87&0.87&0.87&-0.87&0.13\\
1&-1&-1&1&1&-1&-1&1\\
0.37&0.37&1.37&-0.37&-1.37&-1.37&-0.37&1.37\\
1&1&1&1&1&1&1&1\\
\end{bmatrix}
\]
Note that the matrix was to big for me to handle, so there might be some mistakes in calculations, but the approach is important.

Given a unit cube with vertices $x, y, z \in \{-1, 1\}$ and assuming r = -l and t = -b pick appropriate camera frustum parameters r, l, t, b, f, and n to ensure all points of the cube are in the frustum. Orthographic projection is defined as:
\[P=\begin{bmatrix}
\frac{2}{r-1}&0&0&\frac{r+1}{r-1}\\
0&\frac{2}{t-b}&0&\frac{t+b}{t-b}\\
0&0&\frac{2}{f-n}&\frac{f+n}{f-n}\\
0&0&0&1\\
\end{bmatrix}
=\begin{bmatrix}
\frac{1}{r}&0&0&0\\
0&\frac{1}{t}&0&\frac{t+b}{t-b}\\
0&0&\frac{2}{f-n}&\frac{f+n}{f-n}\\
0&0&0&1\\
\end{bmatrix}
\]

\end{document}