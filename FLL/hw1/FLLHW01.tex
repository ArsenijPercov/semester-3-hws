\documentclass[a4paper]{article}
    \usepackage[pdftex]{hyperref}
    \usepackage[latin1]{inputenc}
    \usepackage[english]{babel}
    \usepackage{a4wide}
    \usepackage{amsmath}
    \usepackage{amssymb}
    \usepackage{algorithmic}
    \usepackage{algorithm}
    \usepackage{ifthen}
    \usepackage{listings}
    % move the asterisk at the right position
    \lstset{basicstyle=\ttfamily,tabsize=4,literate={*}{${}^*{}$}1}
    %\lstset{language=C,basicstyle=\ttfamily}
    \usepackage{moreverb}
    \usepackage{palatino}
    \usepackage{multicol}
    \usepackage{tabularx}
    \usepackage{comment}
    \usepackage{verbatim}
    \usepackage{color}
    \usepackage{enumerate}
    \usepackage{enumitem}
    
    %% pdflatex?
    \newif\ifpdf
    \ifx\pdfoutput\undefined
    \pdffalse % we are not running PDFLaTeX
    \else
    \pdfoutput=1 % we are running PDFLaTeX
    \pdftrue
    \fi
    \ifpdf
    \usepackage[pdftex]{graphicx}
    \else
    \usepackage{graphicx}
    \fi
    \ifpdf
    \DeclareGraphicsExtensions{.pdf, .jpg}
    \else
    \DeclareGraphicsExtensions{.eps, .jpg}
    \fi
    
    \parindent=0cm
    \parskip=0cm
    
    \setlength{\columnseprule}{0.4pt}
    \addtolength{\columnsep}{2pt}
    
    \addtolength{\textheight}{5.5cm}
    \addtolength{\topmargin}{-26mm}
    \pagestyle{empty}
    
    %%
    %% Sheet setup
    %% 
    \newcommand{\coursename}{Formal Languages and Logic}
    \newcommand{\courseno}{CO21-320211}
     
    \newcommand{\sheettitle}{Homework}
    \newcommand{\mytitle}{}
    \newcommand{\mytoday}{\textcolor{blue}{September 13}, 2018}
    
    % Current Assignment number
    \newcounter{assignmentno}
    \setcounter{assignmentno}{1}
    
    % Current Problem number, should always start at 1
    \newcounter{problemno}
    \setcounter{problemno}{1}
    
    %%
    %% problem and bonus environment
    %%
    \newcounter{probcalc}
    \newcommand{\problem}[2]{
      \pagebreak[2]
      \setcounter{probcalc}{#2}
      ~\\
      {\large \textbf{Problem \textcolor{blue}{\arabic{assignmentno}}.\textcolor{blue}{\arabic{problemno}}} \hspace{0.2cm}\textit{#1}} \refstepcounter{problemno}\vspace{2pt}\\}
    
    \newcommand{\bonus}[2]{
      \pagebreak[2]
      \setcounter{probcalc}{#2}
      ~\\
      {\large \textbf{Bonus Problem \textcolor{blue}{\arabic{assignmentno}}.\textcolor{blue}{\arabic{problemno}}} \hspace{0.2cm}\textit{#1}} \refstepcounter{problemno}\vspace{2pt}\\}
    
    %% some counters  
    \newcommand{\assignment}{\arabic{assignmentno}}
    
    %% solution  
    \newcommand{\solution}{\pagebreak[2]{\bf Solution:}\\}
    
    %% Hyperref Setup
    \hypersetup{pdftitle={Homework \assignment},
      pdfsubject={\coursename},
      pdfauthor={},
      pdfcreator={},
      pdfkeywords={Formal Languages and Logic},
      %  pdfpagemode={FullScreen},
      %colorlinks=true,
      %bookmarks=true,
      %hyperindex=true,
      bookmarksopen=false,
      bookmarksnumbered=true,
      breaklinks=true,
      %urlcolor=darkblue
      urlbordercolor={0 0 0.7}
    }
    
    \begin{document}
    \coursename \hfill Course: \courseno\\
    Jacobs University Bremen \hfill \mytoday\\
    \textcolor{blue}{Dushan Terzikj and Dragi Kamov}\hfill
    \vspace*{0.3cm}\\
    \begin{center}
    {\Large \sheettitle{} \textcolor{blue}{\assignment}\\}
    \end{center}
    
    \problem{}{0}
    
    \solution
    \textcolor{blue}{
        \begin{enumerate}[label=(\alph*)]
            \item Since $|\Sigma^*|=|\mathbb{N}|$ and $\Sigma^*$ is the set of all words over alphabet $\Sigma=\{1\}$, then the number of words that exist is the number of natural numbers. Analogously, the same goes for $\Sigma=\{a, b\}$.
            \item Since $|\Sigma^n|=|\Sigma|^n$ and $|\Sigma|=k$ we have $k^n$ words of length $n$ over an alphabet of size $k$.
            \item Language is defined as any set made of words over some alphabet $\Sigma$. Let us denote language as $L$. $L \in Pot(\Sigma^*)$. For alphabets $\Sigma_1=\{1\}$ and $\Sigma_2=\{a, b\}$ we got that $|\Sigma_1^*|=|\mathbb{N}|$ and $|\Sigma_2^*|=|\mathbb{N}|$, which means that $|Pot(\mathbb{N})|$ number of languages exist for $\Sigma_1$ and $\Sigma_2$. \\ \\For $\Sigma_3$, where $|\Sigma_3|=k$, we can use the same reasoning as we did for $\Sigma_1$ and $\Sigma_2$. This time we are not looking for $\Sigma_3^n$, but for $\Sigma_3^*$. $|\Sigma_3^*|=|\mathbb{N}| \Rightarrow |Pot(\Sigma_3^*)|=|Pot(\mathbb{N})|$. Using the arguments for $\Sigma_1$ and $\Sigma_2$ we can deduce that $|Pot(\mathbb{N})|$ number of languages exists for alphabet $\Sigma_3$. $Pot(\mathbb{N})=2^{|\mathbb{N}|}$
            \item Since $|\Sigma|=k$ and $|\Sigma^*|=k^n$ we can solve this part using similar reasoning to (c). $|Pot(\Sigma^*)|=2^{k^n}$ in this case. Therefore there are $2^{k^n}$ languages.
            \item Let S be a finite alphabet $S=\{a, b\}$ and $a<b$. These are some of the finite languages on this alphabet:\\
            $\Sigma^0=\{\epsilon\}$\\
            $\Sigma^1=\{a, b\}$\\
            $\Sigma^2=\{aa, ab, ba, bb\}$\\
            $\Sigma^3=\{aaa, aab, aba, abb, baa, bab, bba, bbb\}$\\
            ...\\
            The injective map $i:S\rightarrow\mathbb{N}$ showing from the set of finite languages to $\mathbb{N}$ would look like this:
            \begin{center}
                \begin{tabular}{|c|c|c|c|c|c|c|c|c|c|c|c|c|c|c|c|}
                    \hline
                    \textbf{$\{a,b\}^*$}&\epsilon&a&b&aa&ab&ba&bb&aaa&aab&aba&abb&baa&bab&bba&bbb\\
                    \hline
                    \textbf{$\mathbb{N}$}&0&1&2&3&4&5&6&7&8&9&10&11&12&13&14\\
                    \hline
                \end{tabular}
            \end{center}
        \end{enumerate}
    }
    
    
    \end{document}