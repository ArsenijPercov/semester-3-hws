\documentclass[a4paper]{article}
\usepackage[pdftex]{hyperref}
\usepackage[latin1]{inputenc}
\usepackage[english]{babel}
\usepackage{a4wide}
\usepackage{amsmath}
\usepackage{amssymb}
\usepackage{algorithm}
\usepackage[noend]{algpseudocode}
\usepackage{ifthen}
\usepackage{listings}
% move the asterisk at the right position
\lstset{basicstyle=\ttfamily,tabsize=4,literate={*}{${}^*{}$}1}
%\lstset{language=C,basicstyle=\ttfamily}
\usepackage{moreverb}
\usepackage{palatino}
\usepackage{multicol}
\usepackage{tabularx}
\usepackage{comment}
\usepackage{verbatim}
\usepackage{color}
\usepackage{enumitem}
\usepackage{tikz}
\usetikzlibrary{arrows,shapes.gates.logic.US,shapes.gates.logic.IEC,calc}

\usepackage[left=2cm, right=2cm, top=4cm, bottom=4cm]{geometry}

\usepackage{graphicx}

%% pdflatex?
\newif\ifpdf
\ifx\pdfoutput\undefined
\pdffalse % we are not running PDFLaTeX
\else
\pdfoutput=1 % we are running PDFLaTeX
\pdftrue
\fi
\ifpdf
\usepackage[pdftex]{graphicx}
\else
\usepackage{graphicx}
\fi
\ifpdf
\DeclareGraphicsExtensions{.pdf, .jpg}
\else
\DeclareGraphicsExtensions{.eps, .jpg}
\fi

\parindent=0cm
\parskip=0cm

\setlength{\columnseprule}{0.4pt}
\addtolength{\columnsep}{2pt}

\addtolength{\textheight}{5.5cm}
\addtolength{\topmargin}{-26mm}
\pagestyle{empty}

%%
%% Sheet setup
%% 
\newcommand{\coursename}{Secure and Dependable Systems}
\newcommand{\courseno}{CO21-320203}
 
\newcommand{\sheettitle}{Problem Sheet}
\newcommand{\mytitle}{}
\newcommand{\mytoday}{March 28, 2019}

% Current Assignment number
\newcounter{assignmentno}
\setcounter{assignmentno}{3}

% Current Problem number, should always start at 1
\newcounter{problemno}
\setcounter{problemno}{1}

%%
%% problem and bonus environment
%%
\newcounter{probcalc}
\newcommand{\problem}[2]{
  \pagebreak[2]
  \setcounter{probcalc}{#2}
  ~\\
  {\large \textbf{Problem \arabic{assignmentno}.\arabic{problemno}} \hspace{0.2cm}\textit{#1}} \refstepcounter{problemno}\vspace{2pt}\\}

\newcommand{\bonus}[2]{
  \pagebreak[2]
  \setcounter{probcalc}{#2}
  ~\\
  {\large \textbf{Bonus Problem \textcolor{blue}{\arabic{assignmentno}}.\textcolor{blue}{\arabic{problemno}}} \hspace{0.2cm}\textit{#1}} \refstepcounter{problemno}\vspace{2pt}\\}

%% some counters  
\newcommand{\assignment}{\arabic{assignmentno}}

\makeatletter
\def\BState{\State\hskip-\ALG@thistlm}
\makeatother

%% solution  
\newcommand{\solution}{\pagebreak[2]{\bf Solution:}\\}

%% Hyperref Setup
\hypersetup{pdftitle={Homework \assignment},
  pdfsubject={\coursename},
  pdfauthor={},
  pdfcreator={},
  pdfkeywords={Computer Networks},
  %  pdfpagemode={FullScreen},
  %colorlinks=true,
  %bookmarks=true,
  %hyperindex=true,
  bookmarksopen=false,
  bookmarksnumbered=true,
  breaklinks=true,
  %urlcolor=darkblue
  urlbordercolor={0 0 0.7}
}

\begin{document}
\coursename \hfill Course: \courseno\\
Jacobs University Bremen \hfill \mytoday\\
Dushan Terzikj\hfill
\vspace*{0.3cm}\\
\begin{center}
{\Large \sheettitle{} \assignment\\}
\end{center}

\problem{}{0}
\solution
The question is a little bit ambiguous. In the main problem statement it says $e\prec f$, but it is not clear whether that is a statement or is it just used to explain the relation $\prec$. I would assume the second one.
\begin{enumerate}[label=(\alph*)]
    \item 
    \begin{enumerate}[label=(\roman*)]
        \item \textbf{False}. If $\Theta_L (e)<\Theta_L (f)$ holds, it does not necessarily mean that $e$ occured before $f$. However, $e\prec f$ implies that $\Theta_L (e)<\Theta_L (f)$, but that is not our case.
        \item \textbf{False}. Similar reasoning to (i). The true statement would be $f\prec e\implies\Theta_L(f)<\Theta_L(e)$.
    \end{enumerate}
    \item
    \begin{enumerate}[label=(\roman*)]
    \item It's \textbf{true}, because $e\prec f\iff\Theta_V (e)<\Theta_V (f)$, which means $(e\prec f\implies\Theta_V (e)<\Theta_V (f))\wedge (\Theta_V (e)<\Theta_V (f)\implies e\prec f)$. We know that $\Theta_V (e)<\Theta_V (f)$ holds, so $e\prec f$ must be true.
    \item \textbf{False}. It would have been true if $\Theta_V(f)<\Theta_V(e)$. The reasoning for this is the same as (i).
    \end{enumerate}
    \item If $e$ and $f$ are concurrent, neither of the following statements would hold:
    \begin{itemize}
        \item $\Theta_V(e)<\Theta_V(f)$
        \item $\Theta_V(f)<\Theta_V(e)$
    \end{itemize}
\end{enumerate}

\problem{}{0}
\solution
\begin{enumerate}[label=(\alph*)]
    \item \includegraphics[width=140mm]{SADS1.png}
    \item \includegraphics[width=140mm]{SADS2.png}
    \item $C_1$ is consistent while $C_2$ is not. The reason $C_1$ is consistent is that it fulfills the property of a cut being consistent, i.e., if $a\in C_1$ and $a\prec b$ then $b\prec C_1$. $C_2$ does not show that. For example, if you take $l$ and $s$, $l\in C_2$, while $s\notin C_2$, even though $s\prec l$.
\end{enumerate}

\problem{}{0}
\solution
\begin{algorithm}
\caption{Casual reliable broadcast}\label{euclid}
\begin{algorithmic}
\State
\Procedure{INITIALIZE}{}
\State $rcntDivs:=\perp$ // sequence of messages that were $DELIVER$-ed since the previous $BROADCAST$
\EndProcedure
\State
\Procedure{BROADCAST}{C,m}
\State $BROADCAST(F, rcntDivs.append(m))$ // append $m$ to messages
\State $rcntDivs:=\perp$
\EndProcedure
\State
\State \textbf{upon $DELIVER(m_1, m_2,...m_n)$ do}
\For{$i:=1...n$}
\If{has not previously executed $DELIVER(C,m_i)$}
\State $DELIVER(C, m_i)$
\State $rcntDivs.append(m_i)$
\EndIf
\EndFor
\State
\end{algorithmic}
\end{algorithm}
Reference for this problem:
\href{http://www.sc.ehu.es/acwlaalm/sdi/fault-tolerant-broadcasts.pdf}{http://www.sc.ehu.es/acwlaalm/sdi/fault-tolerant-broadcasts.pdf}.\\ \\
\textbf{See next page!}

\end{document}