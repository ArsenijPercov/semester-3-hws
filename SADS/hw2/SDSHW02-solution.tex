\documentclass[a4paper]{article}
\usepackage[pdftex]{hyperref}
\usepackage[latin1]{inputenc}
\usepackage[english]{babel}
\usepackage{a4wide}
\usepackage{amsmath}
\usepackage{amssymb}
\usepackage{algorithm}
\usepackage[noend]{algpseudocode}
\usepackage{ifthen}
\usepackage{listings}
% move the asterisk at the right position
\lstset{basicstyle=\ttfamily,tabsize=4,literate={*}{${}^*{}$}1}
%\lstset{language=C,basicstyle=\ttfamily}
\usepackage{moreverb}
\usepackage{palatino}
\usepackage{multicol}
\usepackage{tabularx}
\usepackage{comment}
\usepackage{verbatim}
\usepackage{color}
\usepackage{enumitem}
\usepackage{tikz}
\usetikzlibrary{arrows,shapes.gates.logic.US,shapes.gates.logic.IEC,calc}

\usepackage[left=2cm, right=2cm, top=4cm, bottom=4cm]{geometry}

\usepackage{graphicx}

%% pdflatex?
\newif\ifpdf
\ifx\pdfoutput\undefined
\pdffalse % we are not running PDFLaTeX
\else
\pdfoutput=1 % we are running PDFLaTeX
\pdftrue
\fi
\ifpdf
\usepackage[pdftex]{graphicx}
\else
\usepackage{graphicx}
\fi
\ifpdf
\DeclareGraphicsExtensions{.pdf, .jpg}
\else
\DeclareGraphicsExtensions{.eps, .jpg}
\fi

\parindent=0cm
\parskip=0cm

\setlength{\columnseprule}{0.4pt}
\addtolength{\columnsep}{2pt}

\addtolength{\textheight}{5.5cm}
\addtolength{\topmargin}{-26mm}
\pagestyle{empty}

%%
%% Sheet setup
%% 
\newcommand{\coursename}{Secure and Dependable Systems}
\newcommand{\courseno}{CO21-320203}
 
\newcommand{\sheettitle}{Problem Sheet}
\newcommand{\mytitle}{}
\newcommand{\mytoday}{March 14, 2019}

% Current Assignment number
\newcounter{assignmentno}
\setcounter{assignmentno}{2}

% Current Problem number, should always start at 1
\newcounter{problemno}
\setcounter{problemno}{1}

%%
%% problem and bonus environment
%%
\newcounter{probcalc}
\newcommand{\problem}[2]{
  \pagebreak[2]
  \setcounter{probcalc}{#2}
  ~\\
  {\large \textbf{Problem \arabic{assignmentno}.\arabic{problemno}} \hspace{0.2cm}\textit{#1}} \refstepcounter{problemno}\vspace{2pt}\\}

\newcommand{\bonus}[2]{
  \pagebreak[2]
  \setcounter{probcalc}{#2}
  ~\\
  {\large \textbf{Bonus Problem \textcolor{blue}{\arabic{assignmentno}}.\textcolor{blue}{\arabic{problemno}}} \hspace{0.2cm}\textit{#1}} \refstepcounter{problemno}\vspace{2pt}\\}

%% some counters  
\newcommand{\assignment}{\arabic{assignmentno}}

\makeatletter
\def\BState{\State\hskip-\ALG@thistlm}
\makeatother

%% solution  
\newcommand{\solution}{\pagebreak[2]{\bf Solution:}\\}

%% Hyperref Setup
\hypersetup{pdftitle={Homework \assignment},
  pdfsubject={\coursename},
  pdfauthor={},
  pdfcreator={},
  pdfkeywords={Computer Networks},
  %  pdfpagemode={FullScreen},
  %colorlinks=true,
  %bookmarks=true,
  %hyperindex=true,
  bookmarksopen=false,
  bookmarksnumbered=true,
  breaklinks=true,
  %urlcolor=darkblue
  urlbordercolor={0 0 0.7}
}

\begin{document}
\coursename \hfill Course: \courseno\\
Jacobs University Bremen \hfill \mytoday\\
Dushan Terzikj\hfill
\vspace*{0.3cm}\\
\begin{center}
{\Large \sheettitle{} \assignment\\}
\end{center}

\problem{}{0}
\solution\\
\begin{enumerate}[label=(\alph*)]
    \item 
    $\{X=x\wedge N=n\wedge N\ge 0\}\leftarrow$\textbf{Precondition}
    \begin{algorithmic}
    \State $K:=N$
    \State $P:=X$
    \State $Y:=1$
    \While{$K>0$}
    \If{$K\%2=0$}
    \State $P:=P\cdot P$
    \State $K:=K/2$
    \Else
    \State $Y:=Y\cdot P$
    \State $K:=K-1$
    \EndIf{\textbf{fi}}
    \EndWhile{\textbf{od}}
    \end{algorithmic}
    $\{Y=exp(x,n)\}\leftarrow$\textbf{Postcondition}\\
    
    \item 
    $\{X=x\wedge N=n\wedge N\ge 0\}\leftarrow$\textbf{Precondition}
    \begin{algorithmic}
    \State $K:=N$
    \State $P:=X$
    \State $Y:=1$
    \State $\{K=N\wedge P=X\wedge Y=1\}\leftarrow$ \textbf{annotation \#1}
    \While{$K>0$}
    \State $\{Y\cdot exp(P,K)=exp(x,n)\}\leftarrow$ \textbf{annotation \#2 and loop invariant}
    \If{$K\%2=0$}
    \State $P:=P\cdot P$
    \State $K:=K/2$
    \Else
    \State $Y:=Y\cdot P$
    \State $K:=K-1$
    \EndIf{\textbf{fi}}
    \EndWhile{\textbf{od}}
    \end{algorithmic}
    $\{Y=exp(x,n)\}\leftarrow$\textbf{Postcondition}\\
    
    \item 
    \begin{enumerate}[label=(\arabic*)]
        \item From the assignments axioms in line 1, 2 and 3:
        \begin{align*}
            &\{X=x\wedge N=n\wedge N\ge 0\} K:=N; P:=X; Y:=1;\{K=N\wedge P=X\wedge Y=1\}\\
            \Rightarrow &\{X=x\wedge N=n\wedge N\ge 0\} K:=N; P:=X;\{K=N\wedge P=X\wedge 1=1\}\\
            \Rightarrow &\{X=x\wedge N=n\wedge N\ge 0\} K:=N\{K=N\wedge X=X\wedge 1=1\}\\
            \Rightarrow &(X=x\wedge N=n\wedge N\ge 0)\rightarrow (K=K\wedge X=X\wedge 1=1)\\
        \end{align*}
        
        \item From the while loop:
        \begin{align*}
            (K=N\wedge P=X\wedge Y=1)\rightarrow (Y\cdot exp(P,K)=exp(x,n))
        \end{align*}
        
        \item From the while loop:
        \begin{align*}
            (Y\cdot exp(P,K)=exp(x,n)\wedge \neg(K>0))\rightarrow (Y=exp(x,n))
        \end{align*}
        
        \item From the while loop (where $C_1=(P:=P\cdot P; K:=K/2;)$ and\\$C_2=(Y:=Y\cdot P; K:=K-1;)$):
        \begin{align*}
            \{Y\cdot exp(P,K)=exp(x,n)\wedge K>0\}\text{IF }K\%2=0\text{ THEN }C_1\text{ ELSE }C_2\text{ FI }\{Y\cdot exp(P,K)=exp(x,n)\}
        \end{align*}
        Taking the separate commands $C_1$ and $C_2$, we get 2 triplets (the first one is derived here, the second one is derived in (5)).
        \begin{align*}
            &\{Y\cdot exp(P,K)=exp(x,n)\wedge K>0\wedge K\%2=0\}P:=P\cdot P;K=K/2;\{Y\cdot exp(P,K)=exp(x,n)\}\\
            \Rightarrow&\{Y\cdot exp(P,K)=exp(x,n)\wedge K>0\wedge K\%2=0\}P:=P\cdot P\{Y\cdot exp(P,K/2)=exp(x,n)\}\\
            \Rightarrow&(Y\cdot exp(P,K)=exp(x,n)\wedge K>0\wedge K\%2=0)\rightarrow(Y\cdot exp(P\cdot P,K/2)=exp(x,n))\\
        \end{align*}
        
        \item
        \begin{align*}
            &\{Y\cdot exp(P,K)=exp(x,n)\wedge K>0\wedge \neg(K\%2=0)\}Y:=Y\cdot P; K:=K-1;\{Y\cdot exp(P,K)=exp(x,n)\}\\
            \Rightarrow&\{Y\cdot exp(P,K)=exp(x,n)\wedge K>0\wedge \neg(K\%2=0)\}Y:=Y\cdot P\{Y\cdot exp(P,K-1)=exp(x,n)\}\\
            \Rightarrow&(Y\cdot exp(P,K)=exp(x,n)\wedge K>0\wedge \neg(K\%2=0))\rightarrow((Y\cdot P)\cdot exp(P,K-1)=exp(x,n))\\
        \end{align*}
    \end{enumerate}
    
    \item 
    \begin{enumerate}[label=(\arabic*)]
        \item This one is trivial. $(K=K\wedge X=X\wedge 1=1)$ is a tautology, therefore it holds.
        \item Also trivial, but harder to see. Since $Y=1$ we can omit it in the calculations, because all calculations include multiplications and Y will not contribute much. We have that $K=N$ and $P=X$, where $N=n,X=x$. Then we can do the following:
        \begin{align*}
            &exp(P,K)=exp(x,n)\\
            \Rightarrow&exp(X,N)=exp(x,n)\\
            \Rightarrow&exp(x,n)=exp(x,n)\square
        \end{align*}
        \item Since $\neg (K>0)\Rightarrow K=0$. Why? Well:
        \begin{enumerate}[label=(\roman*)]
            \item if K is even then we divide it by 2.
            \item if K is odd then we subtract 2 where we get an even number. Then if $K>2$ we do step (i).
            \item If $K=2$, then $K:=K/2\Rightarrow K=1\Rightarrow K:=K-1\Rightarrow K=0$. This might seem like just one test case, but we can never get any lower than $K=0$.
        \end{enumerate}
        Since $K=0$, then $exp(P,K)=1$, which means $Y=exp(x,n)\square$.
        \item We know that $K>0$ and $K$ is even. Let's denote $exp(P,K)=P^K$ for the sake of beautiful mathematical notation. Then we have $YP^K=x^n$ and $Y(P\cdot P)^{K/2}=x^n$, we can equalize these two equations:
        \begin{align*}
            &YP^K=Y(P\cdot P)^{K/2}\\
            \Rightarrow&P^K=(P\cdot P)^{K/2}\\
            \Rightarrow&P^K=(P^2)^{K/2}\\
            \Rightarrow&P^K=P^{2\cdot K/2}\\
            \Rightarrow&P^K=P^K\square\\
        \end{align*}
        \item This one is similar to (4), only K is odd:
        \begin{align*}
            &YP^K=(YP)P^{K-1}\\
            \Rightarrow&P^K=P\cdot P^{K-1}\\
            \Rightarrow&P^K=P^{K}\square\\
        \end{align*}
    \end{enumerate}
    \newpage
    \item
    $\{X=x\wedge N=n\wedge N\ge 0\}\leftarrow$\textbf{Precondition}
    \begin{algorithmic}
    \State $K:=N$
    \State $P:=X$
    \State $Y:=1$
    \State $\{K=N\wedge P=X\wedge Y=1\}$
    \While{$K>0$}
    \State $\{Y\cdot exp(P,K)=exp(x,n)\}$
    \State [$K$] $\leftarrow$ \textbf{total correctness annotation}
    \If{$K\%2=0$}
    \State $P:=P\cdot P$
    \State $K:=K/2$
    \Else
    \State $Y:=Y\cdot P$
    \State $K:=K-1$
    \EndIf{\textbf{fi}}
    \EndWhile{\textbf{od}}
    \end{algorithmic}
    $\{Y=exp(x,n)\}\leftarrow$\textbf{Postcondition}\\
    
    \item Verification conditions (1), (2) and (3) stay the same. (4) and (5) change a bit or are not used anymore and are replaced by other VCs. Let us derive those:
    \begin{enumerate}[label=(\arabic*)]
        \item 
        \begin{align*}
            (Y\cdot exp(P,K)=exp(x,n)\wedge K>0)\rightarrow (K\ge 0)    
        \end{align*}
        
        \item Now the generated one (or two):
        \begin{align*}
            \{Y\cdot exp(P,K)=exp(x,n)\wedge K>0\wedge K=n\}C\{Y\cdot exp(P,K)=exp(x,n)\wedge K<n\}
        \end{align*}
        Where $C$ is the IF/ELSE statement in our pseudocode. For the IF/ELSE VCs, this is going to be very similar to (c). The difference is that we have 2 new boolean implicants, namely $K=n$ and $K<n$. In this case the $n$ is not the same as the $n$ in the precondition, but I still decided to write it as $n$, because even if it's the same $n$ it will still work. In both ways, $n$ is an arbitrary non-negative integer variable.\\ \\
        We have:
        \begin{align*}
            &\{Y\cdot exp(P,K)=exp(x,n)\wedge K>0\wedge K=n\wedge K\%2=0\}\\&P:=P\cdot P;K=K/2;\\&\{Y\cdot exp(P,K)=exp(x,n)\wedge K<n\}\\
            \Rightarrow&\{Y\cdot exp(P,K)=exp(x,n)\wedge K>0\wedge K=n\wedge K\%2=0\}\\&P:=P\cdot P\\&\{Y\cdot exp(P,K/2)=exp(x,n)\wedge K/2<n\}\\
            \Rightarrow&(Y\cdot exp(P,K)=exp(x,n)\wedge K>0\wedge K=n\wedge K\%2=0)\rightarrow(Y\cdot exp(P\cdot P,K/2)=exp(x,n)\wedge K/2<n)\\
        \end{align*}
        \item Analogously for the second part, we get:
        \begin{align*}
            \Rightarrow&(Y\cdot exp(P,K)=exp(x,n)\wedge K>0\wedge K=n\wedge \neg(K\%2=0))\\&\rightarrow((Y\cdot P)\cdot exp(P,K-1)=exp(x,n)\wedge K-1<n)\\
        \end{align*}
    \end{enumerate}
    \item The proofs for (1), (2), and (3) from (d) are the same. We only need to prove the VCs from (f).
    \begin{enumerate}[label=(\arabic*)]
        \item Trivial: if $K>0$ then $K\ge 0$.
        \item Again trivial: if $K=n$ then:
        \begin{align*}
            \frac{K}{2}&<n\\
            K&<2n
        \end{align*}
        The other part is the same as (4) from (d).
        \item Similarly for this. If $K=n$ then:
        \begin{align*}
            K-1&<n\\
            K&<n+1
        \end{align*}
        The other part is the same as (5) from (d).
    \end{enumerate}
\end{enumerate}


\end{document}