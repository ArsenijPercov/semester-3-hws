\documentclass[a4paper]{article}
\usepackage[pdftex]{hyperref}
\usepackage[latin1]{inputenc}
\usepackage[english]{babel}
\usepackage{a4wide}
\usepackage{amsmath}
\usepackage{amssymb}
\usepackage{algorithmic}
\usepackage{algorithm}
\usepackage{ifthen}
\usepackage{listings}
\usepackage{enumitem}
% move the asterisk at the right position
\lstset{basicstyle=\ttfamily,tabsize=4,literate={*}{${}^*{}$}1}
%\lstset{language=C,basicstyle=\ttfamily}
\usepackage{moreverb}
\usepackage{palatino}
\usepackage{multicol}
\usepackage{tabularx}
\usepackage{comment}
\usepackage{verbatim}
\usepackage{color}

% Because of an error on line 41 I added this
\usepackage{graphicx}

% Used for drawing DFAs and NFAs
\usepackage{tikz}
\usetikzlibrary{automata, positioning}

% Defined checkmark sign
\def\checkmark{\tikz\fill[scale=0.4](0,.35) -- (.25,0) -- (1,.7) -- (.25,.15) -- cycle;}

% Table coloring
\usepackage{color, colortbl}
\usepackage[first=0,last=9]{lcg}

\DeclareFontEncoding{LS1}{}{}
\DeclareFontSubstitution{LS1}{stix}{m}{n}
\DeclareSymbolFont{symbolsstix}{LS1}{stixscr}{m}{n}
\SetSymbolFont{symbolsstix}{bold}{LS1}{stixscr}{b}{n}
\DeclareMathSymbol{\mathvisiblespace}{0}{symbolsstix}{"B6}

% Tab
\newcommand\tab[1][1.15cm]{\hspace*{#1}}

%% pdflatex?
\newif\ifpdf
\ifx\pdfoutput\undefined
\pdffalse % we are not running PDFLaTeX
\else
\pdfoutput=1 % we are running PDFLaTeX
\pdftrue
\fi
%\ifpdf
%\usepackage[pdftex]{graphicx}
%\else
%\usepackage{graphicx}
%\fi
\ifpdf
\DeclareGraphicsExtensions{.pdf, .jpg}
\else
\DeclareGraphicsExtensions{.eps, .jpg}
\fi

\parindent=0cm
\parskip=0cm

\setlength{\columnseprule}{0.4pt}
\addtolength{\columnsep}{2pt}

\addtolength{\textheight}{5.5cm}
\addtolength{\topmargin}{-26mm}
\pagestyle{empty}

%%
%% Sheet setup
%% 
\newcommand{\coursename}{Computability and Complexity}
\newcommand{\courseno}{CO21-320352}
 
\newcommand{\sheettitle}{Homework}
\newcommand{\mytitle}{}
\newcommand{\mytoday}{April 23, 2019}

% Current Assignment number
\newcounter{assignmentno}
\setcounter{assignmentno}{9}

% Current Problem number, should always start at 1
\newcounter{problemno}
\setcounter{problemno}{1}

%%
%% problem and bonus environment
%%
\newcounter{probcalc}
\newcommand{\exercise}[2]{
  \pagebreak[2]
  \setcounter{probcalc}{#2}
  ~\\
  {\large \textbf{Exercise \arabic{problemno}} \hspace{0.2cm}\textit{#1}} \refstepcounter{problemno}\vspace{2pt}\\}

\newcommand{\bonus}[2]{
  \pagebreak[2]
  \setcounter{probcalc}{#2}
  ~\\
  {\large \textbf{Bonus Problem \textcolor{blue}{\arabic{assignmentno}}.\textcolor{blue}{\arabic{problemno}}} \hspace{0.2cm}\textit{#1}} \refstepcounter{problemno}\vspace{2pt}\\}

%% some counters  
\newcommand{\assignment}{\arabic{assignmentno}}

%% solution  
\newcommand{\solution}{\pagebreak[2]{\bf Solution:}\\}

%% Hyperref Setup
\hypersetup{pdftitle={Homework \assignment},
  pdfsubject={\coursename},
  pdfauthor={},
  pdfcreator={},
  pdfkeywords={Computability and Complexity},
  %  pdfpagemode={FullScreen},
  %colorlinks=true,
  %bookmarks=true,
  %hyperindex=true,
  bookmarksopen=false,
  bookmarksnumbered=true,
  breaklinks=true,
  %urlcolor=darkblue
  urlbordercolor={0 0 0.7}
}

\begin{document}
\coursename \hfill Course: \courseno\\
Jacobs University Bremen \hfill \mytoday\\
Dragi Kamov and Dushan Terzikj\hfill
\vspace*{0.3cm}\\
\begin{center}
{\Large \sheettitle{} \assignment\\}
\end{center}

\exercise{}{0}
\solution
Since $A$ is a \textbf{finite} nonempty set, that means that the elements are enumerable.\\ \\
Assume that $A\subseteq \mathcal{A}$, where $\mathcal{A}=\{a_0, a_1 ...\}$. Assume that $\mathcal{A}$ has the property $|\mathcal{A}|=|\mathbb{N}|$. In the this case the elements in A can be enumerated in the following way:
\begin{align*}
    a_0&\rightarrow <a_0>=0\\
    a_1&\rightarrow <a_1>=1\\
    a_2&\rightarrow <a_2>=10\\
    a_3&\rightarrow <a_3>=11\\
    &...\\
    a_n&\rightarrow <a_n>=\text{[binary representation of n]}\\
\end{align*}

A set $A$ can then be encoded into the binary representation of $|A|$ followed by $\#$ which is followed by the encoded elements in A, separated by $\#$. For example $A=\{1, 3\}$:
\begin{equation*}
    <A>=10\# 1\# 11\#
\end{equation*}

Regarding the function $s$, it is not necessary to encode it, but since we are using the power of countable sets, we can encode $<s(a)>=<a>$, but this should be figured out by the TM algorithm.

\exercise{}{0}
\solution
To prove that PARTITION is in NP using a coding scheme consider the following:
\begin{center}
    $ A = \{1,2,3\} $
\end{center}
Consider using the powerset of $ A $:
\begin{center}
    $ P(A) = \{\{\}, \{1\}, \{2\}, \{3\}, \{1, 2\}, \{1, 3\}, \{2, 3\}, \{1, 2, 3\}\} $
\end{center}
Then, using a 2-tape TM: \\
First tape storing the powerset of $ A $ \\
Second tape storing all computations
\begin{center}
    $\implies |P(A)| = 2^{|A|} $ $ NP \leq EXP $
\end{center}

\end{document}