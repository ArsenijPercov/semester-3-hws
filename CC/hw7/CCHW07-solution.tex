\documentclass[a4paper]{article}
\usepackage[pdftex]{hyperref}
\usepackage[latin1]{inputenc}
\usepackage[english]{babel}
\usepackage{a4wide}
\usepackage{amsmath}
\usepackage{amssymb}
\usepackage{algorithmic}
\usepackage{algorithm}
\usepackage{ifthen}
\usepackage{listings}
\usepackage{enumitem}
% move the asterisk at the right position
\lstset{basicstyle=\ttfamily,tabsize=4,literate={*}{${}^*{}$}1}
%\lstset{language=C,basicstyle=\ttfamily}
\usepackage{moreverb}
\usepackage{palatino}
\usepackage{multicol}
\usepackage{tabularx}
\usepackage{comment}
\usepackage{verbatim}
\usepackage{color}

% Because of an error on line 41 I added this
\usepackage{graphicx}

% Used for drawing DFAs and NFAs
\usepackage{tikz}
\usetikzlibrary{automata, positioning}

% Defined checkmark sign
\def\checkmark{\tikz\fill[scale=0.4](0,.35) -- (.25,0) -- (1,.7) -- (.25,.15) -- cycle;}

% Table coloring
\usepackage{color, colortbl}
\usepackage[first=0,last=9]{lcg}

\DeclareFontEncoding{LS1}{}{}
\DeclareFontSubstitution{LS1}{stix}{m}{n}
\DeclareSymbolFont{symbolsstix}{LS1}{stixscr}{m}{n}
\SetSymbolFont{symbolsstix}{bold}{LS1}{stixscr}{b}{n}
\DeclareMathSymbol{\mathvisiblespace}{0}{symbolsstix}{"B6}

% Tab
\newcommand\tab[1][1.15cm]{\hspace*{#1}}

%% pdflatex?
\newif\ifpdf
\ifx\pdfoutput\undefined
\pdffalse % we are not running PDFLaTeX
\else
\pdfoutput=1 % we are running PDFLaTeX
\pdftrue
\fi
%\ifpdf
%\usepackage[pdftex]{graphicx}
%\else
%\usepackage{graphicx}
%\fi
\ifpdf
\DeclareGraphicsExtensions{.pdf, .jpg}
\else
\DeclareGraphicsExtensions{.eps, .jpg}
\fi

\parindent=0cm
\parskip=0cm

\setlength{\columnseprule}{0.4pt}
\addtolength{\columnsep}{2pt}

\addtolength{\textheight}{5.5cm}
\addtolength{\topmargin}{-26mm}
\pagestyle{empty}

%%
%% Sheet setup
%% 
\newcommand{\coursename}{Computability and Complexity}
\newcommand{\courseno}{CO21-320352}
 
\newcommand{\sheettitle}{Homework}
\newcommand{\mytitle}{}
\newcommand{\mytoday}{April 2, 2019}

% Current Assignment number
\newcounter{assignmentno}
\setcounter{assignmentno}{7}

% Current Problem number, should always start at 1
\newcounter{problemno}
\setcounter{problemno}{1}

%%
%% problem and bonus environment
%%
\newcounter{probcalc}
\newcommand{\exercise}[2]{
  \pagebreak[2]
  \setcounter{probcalc}{#2}
  ~\\
  {\large \textbf{Exercise \arabic{problemno}} \hspace{0.2cm}\textit{#1}} \refstepcounter{problemno}\vspace{2pt}\\}

\newcommand{\bonus}[2]{
  \pagebreak[2]
  \setcounter{probcalc}{#2}
  ~\\
  {\large \textbf{Bonus Problem \textcolor{blue}{\arabic{assignmentno}}.\textcolor{blue}{\arabic{problemno}}} \hspace{0.2cm}\textit{#1}} \refstepcounter{problemno}\vspace{2pt}\\}

%% some counters  
\newcommand{\assignment}{\arabic{assignmentno}}

%% solution  
\newcommand{\solution}{\pagebreak[2]{\bf Solution:}\\}

%% Hyperref Setup
\hypersetup{pdftitle={Homework \assignment},
  pdfsubject={\coursename},
  pdfauthor={},
  pdfcreator={},
  pdfkeywords={Computability and Complexity},
  %  pdfpagemode={FullScreen},
  %colorlinks=true,
  %bookmarks=true,
  %hyperindex=true,
  bookmarksopen=false,
  bookmarksnumbered=true,
  breaklinks=true,
  %urlcolor=darkblue
  urlbordercolor={0 0 0.7}
}

\begin{document}
\coursename \hfill Course: \courseno\\
Jacobs University Bremen \hfill \mytoday\\
Dragi Kamov and Dushan Terzikj\hfill
\vspace*{0.3cm}\\
\begin{center}
{\Large \sheettitle{} \assignment\\}
\end{center}

\exercise{}{0}
\solution
To show that \textbf{and true true $=$ true}, we need to define the following $\lambda$-terms (taken from section 7.3 in the LN):
\begin{itemize}
    \item \textbf{true} $\equiv\lambda xy.x$ (=$K!$)
    \item \textbf{false} $\equiv\lambda xy.y$
    \item \textbf{if} $\equiv\lambda pxy.pxy$
    \item \textbf{and} $\equiv\lambda pq.$ \textbf{if} $p$ $q$ \textbf{false}
\end{itemize}
From these, we could further define \textbf{and}:\\ \\
\textbf{and} $\equiv\lambda pq.(\lambda pxy.pxy)pq(\lambda xy.y)$\\ \\
Then we have \textbf{and true true} = $(\lambda pq.(\lambda pxy.pxy)pq(\lambda xy.y))(\lambda xy.x)(\lambda xy.x)$
\begin{align*}
    &(\lambda pq.(\lambda pxy.pxy)pq(\lambda xy.y))(\lambda xy.x)(\lambda xy.x)\\
    \rightarrow^*&(\lambda pq.(\lambda uvw.uvw)pq(\lambda xy.y))(\lambda xy.x)(\lambda xy.x)\\
    \rightarrow^*&(\lambda uvw.uvw)(\lambda xy.x)(\lambda xy.x)(\lambda xy.y)\\
    \rightarrow^*&(\lambda xy.x)(\lambda xy.x)(\lambda xy.y)\\
    \rightarrow^*&(\lambda xy.x)\\
    \rightarrow^*&x\\
\end{align*}
Which in our case means true!\\ \\
\textbf{Second solution:}\\
There is another solution which looks more intuitive and it's easier to understand by a mortal human being:\\ \\
We define \textbf{and} $\equiv\lambda xy.xyF$, where $F$ means false and $T$ means true. \textbf{if} is omitted, because it is trivial and it does not change anything in our current case.
\begin{align*}
    &\text{\textbf{and }}TT\\
    \rightarrow^*&(\lambda xy.xyF)TT\\
    \rightarrow^*&(\lambda y.TyF)T\\
    \rightarrow^*&TTF\\
    \rightarrow^*&(\lambda xy.x)TF\\
    \rightarrow^*&(\lambda y.T)F\\
    \rightarrow^*&T\\
\end{align*}
Reference for the second solution: \textcolor{blue}{\href{http://blog.suspended-chord.info/2012/06/26/csmm---lesson-13-boolean-logic-in-the-lambda-calculus/}{http://blog.suspended-chord.info/2012/06/26/csmm---lesson-13-boolean-logic-in-the-lambda-calculus/}}.

\newpage
\exercise{}{0}
\solution
From the problem sheet \textit{\textbf{Laa = Lbb = Lcc = Lba = Lca = Lcb = false}}, and \textit{\textbf{Lab = Lac = Lbc = true}}. From that we can see that \textbf{L} is true only when $ a < b $, $ b < c $ and $ a < c $. From that we can see that \textbf{L} acts as "properly less than" ordering \textit{a}, \textit{b} and \textit{c} in alphabetical order.

Define: \\
\begin{center}
    SUB $ \equiv \lambda ab.\; b\; - \; a $ \\
    ISZERO $ \equiv \lambda a.\; a(b\; False)\; True $ \\
    LEQ $ \equiv \lambda ab.\; ISZERO(SUB\; a\; b) $ \\
    NOT $ \equiv \lambda a.\; a\; False\; True $ \\
    L $ \equiv \lambda ab. \; NOT(LEQ\; b\; a) $
\end{center}

(Referenced from \textcolor{blue}{\href{https://jwodder.freeshell.org/lambda.html}{https://jwodder.freeshell.org/lambda.html}})

\end{document}