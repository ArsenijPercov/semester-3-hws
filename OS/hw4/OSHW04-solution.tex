\documentclass[a4paper]{article}
\usepackage[pdftex]{hyperref}
\usepackage[latin1]{inputenc}
\usepackage[english]{babel}
\usepackage{a4wide}
\usepackage{amsmath}
\usepackage{amssymb}
\usepackage{algorithmic}
\usepackage{algorithm}
\usepackage{ifthen}
\usepackage{listings}
% move the asterisk at the right position
\lstset{basicstyle=\ttfamily,tabsize=4,literate={*}{${}^*{}$}1}
%\lstset{language=C,basicstyle=\ttfamily}
\usepackage{moreverb}
\usepackage{palatino}
\usepackage{multicol}
\usepackage{tabularx}
\usepackage{comment}
\usepackage{verbatim}
\usepackage{color}
\usepackage{enumitem}
\usepackage{tikz}
\usetikzlibrary{arrows,shapes.gates.logic.US,shapes.gates.logic.IEC,calc}

\usepackage[left=3cm, right=3cm, top=4cm, bottom=4cm]{geometry}

\usepackage{graphicx}

%% pdflatex?
\newif\ifpdf
\ifx\pdfoutput\undefined
\pdffalse % we are not running PDFLaTeX
\else
\pdfoutput=1 % we are running PDFLaTeX
\pdftrue
\fi
\ifpdf
\usepackage[pdftex]{graphicx}
\else
\usepackage{graphicx}
\fi
\ifpdf
\DeclareGraphicsExtensions{.pdf, .jpg}
\else
\DeclareGraphicsExtensions{.eps, .jpg}
\fi

\parindent=0cm
\parskip=0cm

\setlength{\columnseprule}{0.4pt}
\addtolength{\columnsep}{2pt}

\addtolength{\textheight}{5.5cm}
\addtolength{\topmargin}{-26mm}
\pagestyle{empty}

%%
%% Sheet setup
%% 
\newcommand{\coursename}{Operating Systems}
\newcommand{\courseno}{CO20-320202}
 
\newcommand{\sheettitle}{Problem Sheet}
\newcommand{\mytitle}{}
\newcommand{\mytoday}{October 25, 2018}

% Current Assignment number
\newcounter{assignmentno}
\setcounter{assignmentno}{3}

% Current Problem number, should always start at 1
\newcounter{problemno}
\setcounter{problemno}{1}

%%
%% problem and bonus environment
%%
\newcounter{probcalc}
\newcommand{\problem}[2]{
  \pagebreak[2]
  \setcounter{probcalc}{#2}
  ~\\
  {\large \textbf{Problem \arabic{assignmentno}.\arabic{problemno}} \hspace{0.2cm}\textit{#1}} \refstepcounter{problemno}\vspace{2pt}\\}

\newcommand{\bonus}[2]{
  \pagebreak[2]
  \setcounter{probcalc}{#2}
  ~\\
  {\large \textbf{Bonus Problem \textcolor{blue}{\arabic{assignmentno}}.\textcolor{blue}{\arabic{problemno}}} \hspace{0.2cm}\textit{#1}} \refstepcounter{problemno}\vspace{2pt}\\}

%% some counters  
\newcommand{\assignment}{\arabic{assignmentno}}

%% solution  
\newcommand{\solution}{\pagebreak[2]{\bf Solution:}\\}

%% Hyperref Setup
\hypersetup{pdftitle={Homework \assignment},
  pdfsubject={\coursename},
  pdfauthor={},
  pdfcreator={},
  pdfkeywords={Computer Architecture and Programming Languages},
  %  pdfpagemode={FullScreen},
  %colorlinks=true,
  %bookmarks=true,
  %hyperindex=true,
  bookmarksopen=false,
  bookmarksnumbered=true,
  breaklinks=true,
  %urlcolor=darkblue
  urlbordercolor={0 0 0.7}
}

\begin{document}
\coursename \hfill Course: \courseno\\
Jacobs University Bremen \hfill \mytoday\\
Dushan Terzikj\hfill
\vspace*{0.3cm}\\
\begin{center}
{\Large \sheettitle{} \assignment\\}
\end{center}

\problem{}{0}
\solution
\begin{enumerate}[label=\alph*)]
    \item The physical memory has 256 frames. The reason for that is that every page has a fixed size of 1024 bytes which is 1 kilobyte and the physical memory space is 256 kilobytes. 
    \item The logical address space has 16 bits. The reason for that is that the logical address space is limited to maximum of 64 pages: which can be represented by 6 bits, which means in the tuple $(p, d)$ (logical address tuple),  $p=6$. $d=10$, because in the physical space memory each frame has 1024 entries which can be represented with 10 bits.
    \item The physical address space has 18 bits in total. The reason for that is that in the tuple $(f, d)$, $d$ is the same as in (b) and $f=8$, because as we said in (a), the physical memory has 256 frames which can be represented with 8 bits.
    \item We already found the number in (b) $p=6$.
    \item We already found that number $d=10$.
\end{enumerate}

\problem{}{0}
\solution
\begin{enumerate}[label=\alph*)]
    \item
    \begin{tabular}{|c|c|c|c|c|c|c|c|c|c|c|}
        \hline
        reference string&1&2&3&4&1&1&4&2&1&2 \\ \hline
        frame 0&1&1&3&3&1&1&1&1&1&1 \\ \hline
        frame 1&&2&2&4&4&4&4&2&2&2 \\ \hline
        Page fault?&yes&yes&yes&yes&yes&&&yes&& \\ \hline
    \end{tabular}
    \item
    \begin{tabular}{|c|c|c|c|c|c|c|c|c|c|c|}
        \hline
        reference string&1&2&3&4&1&1&4&2&1&2 \\ \hline
        frame 0&1&1&1&4&4&4&4&4&4&4 \\ \hline
        frame 1&&2&2&2*&1&1&1&1&1&1 \\ \hline
        frame 2&&&3&3&3&3&3&2&2&2 \\ \hline
        Page fault?&yes&yes&yes&yes&yes&&&yes&& \\ \hline
    \end{tabular}
    \item
    \begin{tabular}{|c|c|c|c|c|c|c|c|c|c|c|}
        \hline
        reference string&1&2&3&4&1&1&4&2&1&2 \\ \hline
        frame 0&1&1&1&1&1&1&1&1&1&1 \\ \hline
        frame 1&&2&3&4&4&4&4&2&2&2 \\ \hline
        Page fault?&yes&yes&yes&yes&&&&yes&& \\ \hline
    \end{tabular}
    \item
    \begin{tabular}{|c|c|c|c|c|c|c|c|c|c|c|}
        \hline
        reference string&1&2&3&4&1&1&4&2&1&2 \\ \hline
        frame 0&1&1&1&1&1&1&1&1&1&1 \\ \hline
        frame 1&&2&2&2&2&2&2&2&2&2 \\ \hline
        frame 2&&&3&4&4&4&4&4&4&4 \\ \hline
        Page fault?&yes&yes&yes&yes&&&&&& \\ \hline
    \end{tabular}
    \item
    \begin{tabular}{|c|c|c|c|c|c|c|c|c|c|c|}
        \hline
        reference string&1&2&3&4&1&1&4&2&1&2 \\ \hline
        frame 0&1&1&3&3&1&1&1&2&2&2 \\ \hline
        frame 1&&2&2&4&4&4&4&4&1&1 \\ \hline
        Page fault?&yes&yes&yes&yes&yes&&&yes&yes& \\ \hline
    \end{tabular}
    \item
    \begin{tabular}{|c|c|c|c|c|c|c|c|c|c|c|}
        \hline
        reference string&1&2&3&4&1&1&4&2&1&2 \\ \hline
        frame 0&1&1&1&4&4&4&4&4&4&4 \\ \hline
        frame 1&&2&2&2&1&1&1&1&1&1 \\ \hline
        frame 2&&&3&3&3&3&3&2&2&2 \\ \hline
        Page fault?&yes&yes&yes&yes&yes&&&yes&& \\ \hline
    \end{tabular}
\end{enumerate}

\problem{}{0}
\solution
Let me compare both system using the example from problem 2 and let's say $N=3$. In other words, let us say that we have the pages $A=\{1, 2, 3, 4, 1, 1, 4, 2, 1, 2\}$ in this order. Let us denote $t_i$ where $0 \le i \le size(A)-N$ and each $t_i$ corresponds to the page demand time of the respective $A_i$. Using the working-set model we have the following table ($W_{t_{i}}$ is the working-set):\\
\begin{center}
\begin{tabular}{|c|c|}
     \hline
     Time&Working Set  \\ \hline
     $t_0$&1,2,3$\leftarrow$Page fault \\ \hline
     $t_1$&2,3,4$\leftarrow$Page fault \\ \hline
     $t_2$&1,3,4$\leftarrow$Page fault \\ \hline
     $t_3$&1,4$\leftarrow$Page fault \\ \hline
     $t_4$&1,4 \\ \hline
     $t_5$&1,2,4$\leftarrow$Page fault \\ \hline
     $t_6$&1,2,4 \\ \hline
     $t_7$&1,2$\leftarrow$Page fault \\ \hline
\end{tabular}
\end{center}
If we use LRU with 3 pages allocated to each process we will have:\\
\begin{center}
    \begin{tabular}{|c|c|c|c|c|c|c|c|c|c|c|}
        \hline
        reference string&1&2&3&4&1&1&4&2&1&2 \\ \hline
        frame 0&1&1&1&4&4&4&4&4&4&4 \\ \hline
        frame 1&&2&2&2&1&1&1&1&1&1 \\ \hline
        frame 2&&&3&3&3&3&3&2&2&2 \\ \hline
        Page fault?&yes&yes&yes&yes&yes&&&yes&& \\ \hline
    \end{tabular}
\end{center}
In this case we have the same amount of page faults. However, the working-set always discards pages which are not in the interval $[t-N, t)$ but might be used again later. LRU does not do that every time (it might do it sometime if a process is least recently used, but it is used in the next page load).\\ \\
I think LRU is preferable just because of the reasons stated above. \\ \\
Further improvement for the working-set models is that if in the interval $[t-N, t)$ some process which was used in the interval $[t-N-1, t-1)$ is not used anymore should be kept iff the size of the working set is less than $N$.
\end{document}
