\documentclass[a4paper]{article}
    \usepackage[pdftex]{hyperref}
    \usepackage[latin1]{inputenc}
    \usepackage[english]{babel}
    \usepackage{a4wide}
    \usepackage{amsmath}
    \usepackage{amssymb}
    \usepackage{algorithmic}
    \usepackage{algorithm}
    \usepackage{ifthen}
    \usepackage{listings}
    % move the asterisk at the right position
    \lstset{basicstyle=\ttfamily,tabsize=4,literate={*}{${}^*{}$}1}
    %\lstset{language=C,basicstyle=\ttfamily}
    \usepackage{moreverb}
    \usepackage{palatino}
    \usepackage{multicol}
    \usepackage{tabularx}
    \usepackage{comment}
    \usepackage{verbatim}
    \usepackage{color}
    \usepackage{enumitem}
    \usepackage{tikz}
    \usetikzlibrary{arrows,shapes.gates.logic.US,shapes.gates.logic.IEC,calc}
    
    \usepackage[left=3cm, right=3cm, top=4cm, bottom=4cm]{geometry}
    
    \usepackage{graphicx}
    
    %% pdflatex?
    \newif\ifpdf
    \ifx\pdfoutput\undefined
    \pdffalse % we are not running PDFLaTeX
    \else
    \pdfoutput=1 % we are running PDFLaTeX
    \pdftrue
    \fi
    \ifpdf
    \usepackage[pdftex]{graphicx}
    \else
    \usepackage{graphicx}
    \fi
    \ifpdf
    \DeclareGraphicsExtensions{.pdf, .jpg}
    \else
    \DeclareGraphicsExtensions{.eps, .jpg}
    \fi
    
    \parindent=0cm
    \parskip=0cm
    
    \setlength{\columnseprule}{0.4pt}
    \addtolength{\columnsep}{2pt}
    
    \addtolength{\textheight}{5.5cm}
    \addtolength{\topmargin}{-26mm}
    \pagestyle{empty}
    
    %%
    %% Sheet setup
    %% 
    \newcommand{\coursename}{Computer Architecture and Programming Languages}
    \newcommand{\courseno}{CO20-320241}
     
    \newcommand{\sheettitle}{Homework}
    \newcommand{\mytitle}{}
    \newcommand{\mytoday}{\textcolor{blue}{October 23}, 2018}
    
    % Current Assignment number
    \newcounter{assignmentno}
    \setcounter{assignmentno}{6}
    
    % Current Problem number, should always start at 1
    \newcounter{problemno}
    \setcounter{problemno}{1}
    
    %%
    %% problem and bonus environment
    %%
    \newcounter{probcalc}
    \newcommand{\problem}[2]{
      \pagebreak[2]
      \setcounter{probcalc}{#2}
      ~\\
      {\large \textbf{Problem \textcolor{blue}{\arabic{assignmentno}}.\textcolor{blue}{\arabic{problemno}}} \hspace{0.2cm}\textit{#1}} \refstepcounter{problemno}\vspace{2pt}\\}
    
    \newcommand{\bonus}[2]{
      \pagebreak[2]
      \setcounter{probcalc}{#2}
      ~\\
      {\large \textbf{Bonus Problem \textcolor{blue}{\arabic{assignmentno}}.\textcolor{blue}{\arabic{problemno}}} \hspace{0.2cm}\textit{#1}} \refstepcounter{problemno}\vspace{2pt}\\}
    
    %% some counters  
    \newcommand{\assignment}{\arabic{assignmentno}}
    
    %% solution  
    \newcommand{\solution}{\pagebreak[2]{\bf Solution:}\\}
    
    %% Hyperref Setup
    \hypersetup{pdftitle={Homework \assignment},
      pdfsubject={\coursename},
      pdfauthor={},
      pdfcreator={},
      pdfkeywords={Computer Architecture and Programming Languages},
      %  pdfpagemode={FullScreen},
      %colorlinks=true,
      %bookmarks=true,
      %hyperindex=true,
      bookmarksopen=false,
      bookmarksnumbered=true,
      breaklinks=true,
      %urlcolor=darkblue
      urlbordercolor={0 0 0.7}
    }
    
    \begin{document}
    \coursename \hfill Course: \courseno\\
    Jacobs University Bremen \hfill \mytoday\\
    \textcolor{blue}{Dushan Terzikj}\hfill
    \vspace*{0.3cm}\\
    \begin{center}
    {\Large \sheettitle{} \textcolor{blue}{\assignment}\\}
    \end{center}
    
    \problem{}{0}
    \solution
        Because 32-bit architecture machines have 32 registers. With 5 bits you can represent 32 decimal number (ranging from 0 to 31).
    
    \problem{}{0}
    \solution
        \begin{enumerate}[label=(\alph*)]
        \item sub \$t2, \$t1, \$t0
        \item lw \$s2, 4(\$s1)
    \end{enumerate}
    
    \problem{}{0}
    \solution
        \begin{enumerate}[label=(\alph*)]
            \item 
            \begin{tabular}{|c|c|c|c|c|c|}
            \hline
                op&rs&rt&rd&shamt&funct  \\ \hline
                000000&01000&01001&01010&00000&100010 \\ \hline 
            \end{tabular}
            \item
            \begin{tabular}{|c|c|c|c|}
            \hline
                op&rs&rt&const  \\ \hline
                100011&10001&10010&0000000000000100 \\ \hline 
            \end{tabular}
        \end{enumerate}
    
    \problem{}{0}
    \solution
    Basically \$t0 has the decimal value 613566756 and \$t1 has the decimal value 1073217536. Since \$t0 is less that \$t1, register \$t2 will have the value 1, which will result in not jumping to ELSE, but jumping to DONE. Therefore \$t2 will remain 1.
    
    \problem{}{0}
    \solution
    addi \$t0, \$0, 6 \\
    add \$t0, \$t0, \$t0 \\
    add \$t0, \$t0, \$t0 \textbf{$\leftarrow$ get the index of $A[6]$}\\
    add \$t0, \$t0, \$s0 \textbf{$\leftarrow$ get the address of $A[6]$}\\
    lw \$t1, 0(\$t0) \textbf{$\leftarrow$ get the value stored in $A[6]$}\\
    add \$s1, \$s1, \$t1\\
    sw, \$s1, 0(\$t0)
    
    \problem{}{0}
    \solution
    The commands from this problem are referenced from: \href{https://courses.cs.washington.edu/courses/cse378/01au/files/pdf/378-ln8.pdf}{here}.\\
    lui \$t0 0000 0000 0010 0011 \textbf{$\leftarrow$ add the most significant 16 bits}\\
    ori \$t0, \$t0, 0000 0000 0010 0011 \textbf{$\leftarrow$ add the least significant 16 bits}\\
    sw \$t0, 0(\$s4)
    
    \newpage
    
    \problem{}{0}
    \solution
    \begin{align*}
        &\text{add \$t0, \$0, \$0}\\
        \text{LOOP: }&\text{slti \$t1, \$t0, 8}\\
        &\text{beq \$t1, \$s0, EXIT}\\
        &\text{addi \$s0, \$s0, 3}\\
        &\text{addi, \$t0, \$t0, 1}\\
        &\text{j LOOP}\\
        \text{EXIT: }&
    \end{align*}
    
    \end{document}