\documentclass[a4paper]{article}
\usepackage[pdftex]{hyperref}
\usepackage[latin1]{inputenc}
\usepackage[english]{babel}
\usepackage{a4wide}
\usepackage{amsmath}
\usepackage{amssymb}
\usepackage{algorithmic}
\usepackage{algorithm}
\usepackage{ifthen}
\usepackage{listings}
\usepackage{xcolor}
% move the asterisk at the right position
\lstset{basicstyle=\ttfamily,tabsize=4,literate={*}{${}^*{}$}1}
%\lstset{language=C,basicstyle=\ttfamily}
\usepackage{moreverb}
\usepackage{palatino}
\usepackage{multicol}
\usepackage{tabularx}
\usepackage{comment}
\usepackage{verbatim}
\usepackage{color}
\usepackage{enumitem}
\usepackage{tikz}
\usetikzlibrary{arrows,shapes.gates.logic.US,shapes.gates.logic.IEC,calc}

\usepackage[left=3cm, right=3cm, top=4cm, bottom=4cm]{geometry}

\usepackage{graphicx}

%% pdflatex?
\newif\ifpdf
\ifx\pdfoutput\undefined
\pdffalse % we are not running PDFLaTeX
\else
\pdfoutput=1 % we are running PDFLaTeX
\pdftrue
\fi
\ifpdf
\usepackage[pdftex]{graphicx}
\else
\usepackage{graphicx}
\fi
\ifpdf
\DeclareGraphicsExtensions{.pdf, .jpg}
\else
\DeclareGraphicsExtensions{.eps, .jpg}
\fi

\parindent=0cm
\parskip=0cm

\setlength{\columnseprule}{0.4pt}
\addtolength{\columnsep}{2pt}

\addtolength{\textheight}{5.5cm}
\addtolength{\topmargin}{-26mm}
\pagestyle{empty}

%%
%% Sheet setup
%% 
\newcommand{\coursename}{Computer Architecture and Programming Languages}
\newcommand{\courseno}{CO20-320241}
 
\newcommand{\sheettitle}{Homework}
\newcommand{\mytitle}{}
\newcommand{\mytoday}{November 13, 2018}

% Current Assignment number
\newcounter{assignmentno}
\setcounter{assignmentno}{8}

% Current Problem number, should always start at 1
\newcounter{problemno}
\setcounter{problemno}{1}

%%
%% problem and bonus environment
%%
\newcounter{probcalc}
\newcommand{\problem}[2]{
  \pagebreak[2]
  \setcounter{probcalc}{#2}
  ~\\
  {\large \textbf{Problem \arabic{assignmentno}.\arabic{problemno}} \hspace{0.2cm}\textit{#1}} \refstepcounter{problemno}\vspace{2pt}\\}

\newcommand{\bonus}[2]{
  \pagebreak[2]
  \setcounter{probcalc}{#2}
  ~\\
  {\large \textbf{Bonus Problem \arabic{assignmentno}.\arabic{problemno}} \hspace{0.2cm}\textit{#1}} \refstepcounter{problemno}\vspace{2pt}\\}

%% some counters  
\newcommand{\assignment}{\arabic{assignmentno}}

%% solution  
\newcommand{\solution}{\pagebreak[2]{\bf Solution:}\\}

%% Hyperref Setup
\hypersetup{pdftitle={Homework \assignment},
  pdfsubject={\coursename},
  pdfauthor={},
  pdfcreator={},
  pdfkeywords={Computer Architecture and Programming Languages},
  %  pdfpagemode={FullScreen},
  %colorlinks=true,
  %bookmarks=true,
  %hyperindex=true,
  bookmarksopen=false,
  bookmarksnumbered=true,
  breaklinks=true,
  %urlcolor=darkblue
  urlbordercolor={0 0 0.7}
}

\definecolor{mGreen}{rgb}{0,0.6,0}
\definecolor{mGray}{rgb}{0.5,0.5,0.5}
\definecolor{mPurple}{rgb}{0.58,0,0.82}
\definecolor{backgroundColour}{rgb}{0.95,0.95,0.92}

\lstdefinestyle{CStyle}{
    backgroundcolor=\color{backgroundColour},   
    commentstyle=\color{mGreen},
    keywordstyle=\color{magenta},
    numberstyle=\tiny\color{mGray},
    stringstyle=\color{mPurple},
    basicstyle=\footnotesize,
    breakatwhitespace=false,         
    breaklines=true,                 
    captionpos=b,                    
    keepspaces=true,                 
    numbers=left,                    
    numbersep=5pt,                  
    showspaces=false,                
    showstringspaces=false,
    showtabs=false,                  
    tabsize=2,
    language=C
}

\begin{document}
\coursename \hfill Course: \courseno\\
Jacobs University Bremen \hfill \mytoday\\
Dushan Terzikj\hfill
\vspace*{0.3cm}\\
\begin{center}
{\Large \sheettitle{} \assignment}\\
\end{center}

\problem{}{0}
\solution
\begin{enumerate}
    \item[(\textit{i})] $0.78125\times 2=1.5625\rightarrow 1$\\
    $0.5625\times 2=1.125\rightarrow 1$\\
    $0.125\times 2=1.25\rightarrow 0$\\
    $0.25\times 2=0.5\rightarrow 0$\\
    $0.5\times 2=1.0\rightarrow 1$\\
    $\Rightarrow 0.11001\Rightarrow1.1001\times 2^{-1}$\\
    $\Rightarrow exp=-1\Rightarrow 127-1=126_{10}=01111110_{2}$\\
    \begin{tabular}{c|c|c}
        sign&exponent&fraction  \\ \hline
        0&01111110&10010000000000000000000\\ 
    \end{tabular}
    \item[(\textit{ii})] $27\div 2=13\xrightarrow{\text{remainder}}1$\\
    $13\div 2=6\xrightarrow{\text{remainder}}1$\\
    $6\div 2=3\xrightarrow{\text{remainder}}0$\\
    $3\div 2=1\xrightarrow{\text{remainder}}1$\\
    $1\div 2=0\xrightarrow{\text{remainder}}1$\\
    $\Rightarrow 11011=1.1011\times 2^4$\\
    $\Rightarrow exp=4+127=131_{10}=10000011_2$\\
    $0.3515625\times 2=0.703125\rightarrow 0$\\
    $0.703125\times 2=1.40625\rightarrow 1$\\
    $0.40625\times 2=0.8125\rightarrow 0$\\
    $0.8125\times 2=1.625\rightarrow 1$\\
    $0.625\times 2=1.25\rightarrow 1$\\
    $0.25\times 2=0.5\rightarrow 0$\\
    $0.5\times 2=1.0\rightarrow 1$\\ \\
    \begin{tabular}{c|c|c}
        sign&exponent&fraction  \\ \hline
        0&10000011&01011010000000000000000\\ 
    \end{tabular}
\end{enumerate}

\problem{}{0}
\solution
\begin{enumerate}
    \item MIPS has an alignment restriction, that means words must start at addresses that are multiples of 4. \textbf{true}
    \item All MIPS instructions are 30 bits long. \textbf{false}
    \item MIPS can perform arithmetic operations on memory locations. \textbf{false}
    \item Parameters to functions are always passed via the stack. \textbf{false}
    \item A procedure jumps to the address stored in the stack pointer register after it finishes execution. \textbf{false}
\end{enumerate}

\problem{}{0}
\solution
\lstinputlisting{task8-3.s}
\newpage
\problem{}{0}
\solution
\begin{enumerate}[label=(\alph*)]
    \item 26 bits.
    \item One alternative is to use \textit{jr}. In order to use this we can store the address in a register (say \textit{rd}) and then use \textit{jr rd}.\\ \\
    Another alternative is to use \textit{jal}. This works in a way that the first four digits of the program counter are contacenated with the 26-bit address (which is shifted to the left by 2 positions).
\end{enumerate}

\problem{}{0}
\solution\\
\begin{tabular}{|c|c|c|c|c|c|}
    \hline
    Class&CPI on P1&CPI on P2&Freq&CPI on P1$\times$Freq&CPI on P2$\times$Freq  \\ \hline
    A&1&2&60\%&0.6&1.2  \\ \hline
    B&2&2&10\%&0.2&0.2 \\ \hline
    C&3&2&10\%&0.3&0.2  \\ \hline
    D&4&4&10\%&0.4&0.4  \\ \hline
    E&3&4&10\%&0.3&0.4  \\ \hline
    &&&&$\Sigma =1.8$&$\Sigma =2.4$ \\ \hline
\end{tabular}\\ \\
$\text{CPU time for P1}=\frac{13\times 1.8}{4}=5.85$ \# according to the formula in the slides\\
$\text{CPU time for P2}=\frac{14\times 2.4}{6}=5.6$ \# according to the formula in the slides\\
13 is the total number of instructions for P1 and 14 is the total number of instructions for P2. 4 is the clock rate of P1 and 6 is the clock rate of P2.\\ \\
With these calculations we see that P2 is faster than P1 by a factor $\frac{5.85}{5.6}=1.04464286$

\problem{}{0}
\solution\\
\begin{tabular}{|c|c|c|}
    \hline
    Class&CPI on P1&CPI on P2  \\ \hline
    A&1&2 \\ \hline
    B&3&3 \\ \hline
    C&3&2 \\ \hline
    D&4&3 \\ \hline
    E&2&3 \\ \hline
\end{tabular}\\ \\
Since A shows up twice as much as the others we have 6 classes to include in the calculation of the Average CPI of P1 and P2.
\begin{align*}
    P1\_CPI\_avg=\frac{2\times 1+3+3+4+2}{6}=2.33
\end{align*}
\begin{align*}
    P2\_CPI\_avg=\frac{2\times 2+3+2+3+3}{6}=2.5
\end{align*}
The performance ratio is the following:
\begin{align*}
    r=\frac{performance\_P2}{performance\_P1}=\frac{4}{2}\times\frac{2.33}{2.5}=1.86400
\end{align*}
This means that P2 is 86.4\% faster than P1. 
\end{document}
