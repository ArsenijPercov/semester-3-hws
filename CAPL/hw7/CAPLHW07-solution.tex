\documentclass[a4paper]{article}
\usepackage[pdftex]{hyperref}
\usepackage[latin1]{inputenc}
\usepackage[english]{babel}
\usepackage{a4wide}
\usepackage{amsmath}
\usepackage{amssymb}
\usepackage{algorithmic}
\usepackage{algorithm}
\usepackage{ifthen}
\usepackage{listings}
\usepackage{xcolor}
% move the asterisk at the right position
\lstset{basicstyle=\ttfamily,tabsize=4,literate={*}{${}^*{}$}1}
%\lstset{language=C,basicstyle=\ttfamily}
\usepackage{moreverb}
\usepackage{palatino}
\usepackage{multicol}
\usepackage{tabularx}
\usepackage{comment}
\usepackage{verbatim}
\usepackage{color}
\usepackage{enumitem}
\usepackage{tikz}
\usetikzlibrary{arrows,shapes.gates.logic.US,shapes.gates.logic.IEC,calc}

\usepackage[left=3cm, right=3cm, top=4cm, bottom=4cm]{geometry}

\usepackage{graphicx}

%% pdflatex?
\newif\ifpdf
\ifx\pdfoutput\undefined
\pdffalse % we are not running PDFLaTeX
\else
\pdfoutput=1 % we are running PDFLaTeX
\pdftrue
\fi
\ifpdf
\usepackage[pdftex]{graphicx}
\else
\usepackage{graphicx}
\fi
\ifpdf
\DeclareGraphicsExtensions{.pdf, .jpg}
\else
\DeclareGraphicsExtensions{.eps, .jpg}
\fi

\parindent=0cm
\parskip=0cm

\setlength{\columnseprule}{0.4pt}
\addtolength{\columnsep}{2pt}

\addtolength{\textheight}{5.5cm}
\addtolength{\topmargin}{-26mm}
\pagestyle{empty}

%%
%% Sheet setup
%% 
\newcommand{\coursename}{Computer Architecture and Programming Languages}
\newcommand{\courseno}{CO20-320241}
 
\newcommand{\sheettitle}{Homework}
\newcommand{\mytitle}{}
\newcommand{\mytoday}{November 6, 2018}

% Current Assignment number
\newcounter{assignmentno}
\setcounter{assignmentno}{7}

% Current Problem number, should always start at 1
\newcounter{problemno}
\setcounter{problemno}{1}

%%
%% problem and bonus environment
%%
\newcounter{probcalc}
\newcommand{\problem}[2]{
  \pagebreak[2]
  \setcounter{probcalc}{#2}
  ~\\
  {\large \textbf{Problem \arabic{assignmentno}.\arabic{problemno}} \hspace{0.2cm}\textit{#1}} \refstepcounter{problemno}\vspace{2pt}\\}

\newcommand{\bonus}[2]{
  \pagebreak[2]
  \setcounter{probcalc}{#2}
  ~\\
  {\large \textbf{Bonus Problem \arabic{assignmentno}.\arabic{problemno}} \hspace{0.2cm}\textit{#1}} \refstepcounter{problemno}\vspace{2pt}\\}

%% some counters  
\newcommand{\assignment}{\arabic{assignmentno}}

%% solution  
\newcommand{\solution}{\pagebreak[2]{\bf Solution:}\\}

%% Hyperref Setup
\hypersetup{pdftitle={Homework \assignment},
  pdfsubject={\coursename},
  pdfauthor={},
  pdfcreator={},
  pdfkeywords={Computer Architecture and Programming Languages},
  %  pdfpagemode={FullScreen},
  %colorlinks=true,
  %bookmarks=true,
  %hyperindex=true,
  bookmarksopen=false,
  bookmarksnumbered=true,
  breaklinks=true,
  %urlcolor=darkblue
  urlbordercolor={0 0 0.7}
}

\definecolor{mGreen}{rgb}{0,0.6,0}
\definecolor{mGray}{rgb}{0.5,0.5,0.5}
\definecolor{mPurple}{rgb}{0.58,0,0.82}
\definecolor{backgroundColour}{rgb}{0.95,0.95,0.92}

\lstdefinestyle{CStyle}{
    backgroundcolor=\color{backgroundColour},   
    commentstyle=\color{mGreen},
    keywordstyle=\color{magenta},
    numberstyle=\tiny\color{mGray},
    stringstyle=\color{mPurple},
    basicstyle=\footnotesize,
    breakatwhitespace=false,         
    breaklines=true,                 
    captionpos=b,                    
    keepspaces=true,                 
    numbers=left,                    
    numbersep=5pt,                  
    showspaces=false,                
    showstringspaces=false,
    showtabs=false,                  
    tabsize=2,
    language=C
}

\begin{document}
\coursename \hfill Course: \courseno\\
Jacobs University Bremen \hfill \mytoday\\
Dushan Terzikj\hfill
\vspace*{0.3cm}\\
\begin{center}
{\Large \sheettitle{} \assignment}\\
\end{center}

\problem{}{0}
\solution
\lstinputlisting{task-1.s}

\problem{}{0}
\solution
\lstinputlisting{task-2.s}

\problem{}{0}
\solution
\begin{lstlisting}[style=CStyle]
int i = 0; // this is bascially the \$s3 register
while(A[i] != -1){
    i++
}
\end{lstlisting}
\newpage
\problem{}{0}
\solution
\begin{center}
    \begin{tabular}{|c|c|c|}
        \hline
        PC&Machine Code&Binary Machine Code \\ \hline
        60000&0 0 19 9 2 0&000000 00000 10011 01001 00010 000000 \\ \hline
        60004&0 9 22 9 0 32& 000000 01001 10110 01001 00000 100000 \\ \hline
        60008&35 9 8 0&100011 01001 01000 0000000000000000  \\ \hline
        60012&4 8 21 2&000100 01000 10101 0000000000000010  \\ \hline
        60016&8 19 19 1&001000 10011 10011 0000000000000001  \\ \hline
        60020&2 15000&000010 00000000000011101010011000 \\ \hline
        60024&&  \\ \hline
    \end{tabular}
\end{center}

\problem{}{0}
\solution
\begin{enumerate}[label=(\alph*)]
    \item $0C000000_{16}=0000 1100 0000 0000 0000 0000 0000 0000_{2}$ (Please do not make me write the calculations, it's obvious that this is the number, note that every segment of 4 bits is 0, except the second segment which is 12 in decimal and C in hex).
    We can see that it is a positive number, therefore the number in decimal is 201326592 ($201326592=2^{26}+2^{27}$). \\
    Analogously we do the same for $C4630000_{16}=1100 0100 0110 0011 0000 0000 0000 0000_{2}$. This number is negative so we have to convert it to positive by first inverting all bits and then adding one:\\ \\
    $1100 0100 0110 0011 0000 0000 0000 0000_{2}\xrightarrow{\text{invert bits}}0011 1011 1001 1100 1111 1111 1111 1111$\\
    $0011 1011 1001 1100 1111 1111 1111 1111\xrightarrow{\text{add 1}}0011 1011 1001 1101 0000 0000 0000 0000$\\ \\
    $0011 1011 1001 1101 0000 0000 0000 0000_2=1000144896_{10}$ which means that:\\
    $1100 0100 0110 0011 0000 0000 0000 0000_{2}=-1000144896_{10}$
    \item For the first hex number it is the same as (a) since it is a positive number. For the second number:\\ \\
    $C4630000_{16}=3294822400_{10}$\\ \\
    This is how I got the number:\\ \\
    $C4630000_{16}=11000100011000110000000000000000_{2}$\\
    $2^{16}+2^{17}+2^{21}+2^{22}+2^{26}+2^{30}+2^{31}=3294822400_{10}$
    \item For the first number it would have been shift right (\textit{sra}).\\ \\
    For the second number:\\
    110001 00011 00011 0000000000000000\\ \\
    It would not mean anything because that opcode does not exist in the MIPS documentation.
\end{enumerate}

\end{document}
