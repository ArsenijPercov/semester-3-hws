\documentclass[a4paper]{article}
    \usepackage[pdftex]{hyperref}
    \usepackage[latin1]{inputenc}
    \usepackage[english]{babel}
    \usepackage{a4wide}
    \usepackage{amsmath}
    \usepackage{amssymb}
    \usepackage{algorithmic}
    \usepackage{algorithm}
    \usepackage{ifthen}
    \usepackage{listings}
    % move the asterisk at the right position
    \lstset{basicstyle=\ttfamily,tabsize=4,literate={*}{${}^*{}$}1}
    %\lstset{language=C,basicstyle=\ttfamily}
    \usepackage{moreverb}
    \usepackage{palatino}
    \usepackage{multicol}
    \usepackage{tabularx}
    \usepackage{comment}
    \usepackage{verbatim}
    \usepackage{color}
    \usepackage{enumitem}
    \usepackage{tikz}
    \usetikzlibrary{arrows,shapes.gates.logic.US,shapes.gates.logic.IEC,calc}
    
    \usepackage[left=3cm, right=3cm, top=4cm, bottom=4cm]{geometry}
    
    \usepackage{graphicx}
    
    %% pdflatex?
    \newif\ifpdf
    \ifx\pdfoutput\undefined
    \pdffalse % we are not running PDFLaTeX
    \else
    \pdfoutput=1 % we are running PDFLaTeX
    \pdftrue
    \fi
    \ifpdf
    \usepackage[pdftex]{graphicx}
    \else
    \usepackage{graphicx}
    \fi
    \ifpdf
    \DeclareGraphicsExtensions{.pdf, .jpg}
    \else
    \DeclareGraphicsExtensions{.eps, .jpg}
    \fi
    
    \parindent=0cm
    \parskip=0cm
    
    \setlength{\columnseprule}{0.4pt}
    \addtolength{\columnsep}{2pt}
    
    \addtolength{\textheight}{5.5cm}
    \addtolength{\topmargin}{-26mm}
    \pagestyle{empty}
    
    %%
    %% Sheet setup
    %% 
    \newcommand{\coursename}{Computer Architecture and Programming Languages}
    \newcommand{\courseno}{CO20-320241}
     
    \newcommand{\sheettitle}{Homework}
    \newcommand{\mytitle}{}
    \newcommand{\mytoday}{\textcolor{blue}{October 9}, 2018}
    
    % Current Assignment number
    \newcounter{assignmentno}
    \setcounter{assignmentno}{4}
    
    % Current Problem number, should always start at 1
    \newcounter{problemno}
    \setcounter{problemno}{1}
    
    %%
    %% problem and bonus environment
    %%
    \newcounter{probcalc}
    \newcommand{\problem}[2]{
      \pagebreak[2]
      \setcounter{probcalc}{#2}
      ~\\
      {\large \textbf{Problem \textcolor{blue}{\arabic{assignmentno}}.\textcolor{blue}{\arabic{problemno}}} \hspace{0.2cm}\textit{#1}} \refstepcounter{problemno}\vspace{2pt}\\}
    
    \newcommand{\bonus}[2]{
      \pagebreak[2]
      \setcounter{probcalc}{#2}
      ~\\
      {\large \textbf{Bonus Problem \textcolor{blue}{\arabic{assignmentno}}.\textcolor{blue}{\arabic{problemno}}} \hspace{0.2cm}\textit{#1}} \refstepcounter{problemno}\vspace{2pt}\\}
    
    %% some counters  
    \newcommand{\assignment}{\arabic{assignmentno}}
    
    %% solution  
    \newcommand{\solution}{\pagebreak[2]{\bf Solution:}\\}
    
    %% Hyperref Setup
    \hypersetup{pdftitle={Homework \assignment},
      pdfsubject={\coursename},
      pdfauthor={},
      pdfcreator={},
      pdfkeywords={Computer Architecture and Programming Languages},
      %  pdfpagemode={FullScreen},
      %colorlinks=true,
      %bookmarks=true,
      %hyperindex=true,
      bookmarksopen=false,
      bookmarksnumbered=true,
      breaklinks=true,
      %urlcolor=darkblue
      urlbordercolor={0 0 0.7}
    }
    
    \begin{document}
    \coursename \hfill Course: \courseno\\
    Jacobs University Bremen \hfill \mytoday\\
    \textcolor{blue}{Dushan Terzikj}\hfill
    \vspace*{0.3cm}\\
    \begin{center}
    {\Large \sheettitle{} \textcolor{blue}{\assignment}\\}
    \end{center}
    
    \problem{}{0}
    \solution
    \textcolor{blue}{
        The algebraic expression for the logic circuit is the following:
        \begin{align}
            (A\oplus B)\cdot \overline{(B\oplus C)}\cdot C=X
        \end{align}
        And here is the truth table for the same expression:
        \begin{center}
            \begin{tabular}{|c|c|c|c|}
            \hline
            A&B&C&X  \\ \hline
            0&0&0&0  \\ \hline
            0&0&1&0  \\ \hline
            0&1&0&0  \\ \hline
            0&1&1&1  \\ \hline
            1&0&0&0  \\ \hline
            1&0&1&0  \\ \hline
            1&1&0&0  \\ \hline
            1&1&1&0  \\ \hline
            \end{tabular}
        \end{center}
        According the to table, the following inputs is necessary for $X=1$:
        \begin{align*}
            A=0,\text{ }&B=C=1
        \end{align*}
    }
    
    \problem{}{0}
    \solution
    \textcolor{blue}{
        \begin{enumerate}[label=(\alph*)]
            \item 
            \begin{equation}
                \overline{(A\cdot B+C)}\oplus ((B+C)\cdot\overline{A})=Y
            \end{equation}
            \begin{center}
                \begin{tabular}{|c|c|c|c|}
                    \hline
                    A&B&C&Y  \\ \hline
                    0&0&0&1  \\ \hline
                    0&0&1&1  \\ \hline
                    0&1&0&0  \\ \hline
                    0&1&1&1  \\ \hline
                    1&0&0&1  \\ \hline
                    1&0&1&0  \\ \hline
                    1&1&0&0  \\ \hline
                    1&1&1&0  \\ \hline
                \end{tabular}
            \end{center}
            \item
            \begin{align*}
                \overline{A}\cdot\overline{B}\cdot\overline{C}&+\overline{A}\cdot\overline{B}\cdot C+\overline{A}\cdot B\cdot C+A\cdot\overline{B}\cdot\overline{C}\\
                &=\overline{A}\cdot\overline{B}\cdot(\overline{C}+C)+\overline{A}\cdot B\cdot C+A\cdot\overline{B}\cdot\overline{C}\\
                &=\overline{A}\cdot\overline{B}+\overline{A}\cdot B\cdot C+A\cdot\overline{B}\cdot\overline{C}
            \end{align*}
        \end{enumerate}
    }
    
    \problem{}{0}
    \solution
    \textcolor{blue}{
        \begin{enumerate}[label=(\alph*)]
            \item $27_{10}=00011011_2$
            \item $66_{10}=01000010_2$
            \item $-18_{10}=00010010_2\xrightarrow[]{\text{invert bits}}11101101_2\xrightarrow[]{\text{add 1}}11101110_2$
            \item $127_{10}=01111111_2$
            \item $-127_{10}=01111111_2\xrightarrow[]{\text{invert bits}}1000000_2\xrightarrow[]{\text{add 1}}10000001_2$
            \item $-128_{10}=10000000_2\xrightarrow[]{\text{invert bits}}01111111_2\xrightarrow[]{\text{add 1}}10000000_2$
            \item $131_{10}=10000011_2$
            \item $-7_{10}=00000111_2\xrightarrow[]{\text{invert bits}}11111000_2\xrightarrow[]{\text{add 1}}11111001_2$
        \end{enumerate}
    }
    
    \problem{}{0}
    \solution
    \textcolor{blue}{
        \begin{enumerate}[label=(\alph*)]
            \item $00011000_2=24_{10}$
            \item $11110101_2\xrightarrow{\text{invert bits}}00001010_2\xrightarrow{\text{add 1}}00001011_2=-11_{10}$
            \item $01011011_2=91_{10}$
            \item $10110110_2\xrightarrow{\text{invert bits}}01001001_2\xrightarrow{\text{add 1}}01001010_2=-74_{10}$
            \item $11111111_2\xrightarrow{\text{invert bits}}00000000_2\xrightarrow{\text{add 1}}00000001_2=-1_{10}$
            \item $01101111_2=111_{10}$
            \item $10000001_2\xrightarrow{\text{invert bits}}01111110_2\xrightarrow{\text{add 1}}01111111_2=-127_{10}$
            \item $10000000_2\xrightarrow{\text{invert bits}}01111111_2\xrightarrow{\text{add 1}}10000000_2=-128_{10}$
        \end{enumerate}
    }
    
    \problem{}{0}
    \solution
    \textcolor{blue}{
        \begin{enumerate}[label=(\alph*)]
            \item $27_{10}=00100111_{\text{BCD}}$\\
            $36_{10}=00110110_{\text{BCD}}$ \\ \\
            \begin{tabular}{ccc}
                &$00100111$&  \\
                +&$00110110$& \\ \hline
                &$01011101$&$\leftarrow$Invalid code groups, add $6$ \\
                +&$\text{****}0110$& \\ \hline
                &$01100011$&
            \end{tabular}\\ \\
            $01100011_{\text{BCD}}=63_{10}$
            \item $73_{10}=01110011_{\text{BCD}}$\\
            $29_{10}=00101001_{\text{BCD}}$ \\ \\
            \begin{tabular}{ccc}
                &$01110011$&  \\
                +&$0010100$& \\ \hline
                &$10011100$&$\leftarrow$Invalid code groups, add $6$ \\
                +&$\text{****}0110$& \\ \hline
                &$10100010$&$\leftarrow$Invalid code groups, add $6$ \\
                +&$0110\text{****}$& \\ \hline
                &$000100000010$&
            \end{tabular}\\ \\
            $000100000010_{\text{BCD}}=102_{10}$
        \end{enumerate}
    }
    \newpage
    
    \problem{}{0}
    \solution
    \textcolor{blue}{
        \begin{enumerate}[label=(\alph*)]
            \item Since all 8 bits are used for magnitude the range is $[0, 2^8-1]$.
            \item Since one bit is used for sign, the range is $[-2^7, 2^7-1]$.
            \item Like (a) the range is $[0, 2^{11}-1]$.
            \item Like (b) the range is $[-2^{10}, 2^{10}-1]$.
            \item Like (b) and (d) the range is $[-2^{15}, 2^{15}-1]$.
        \end{enumerate}
    }
    
    \end{document}