\documentclass[a4paper]{article}
\usepackage[pdftex]{hyperref}
\usepackage[latin1]{inputenc}
\usepackage[english]{babel}
\usepackage{a4wide}
\usepackage{amsmath}
\usepackage{amssymb}
\usepackage{algorithmic}
\usepackage{algorithm}
\usepackage{ifthen}
\usepackage{listings}
\usepackage{xcolor}
% move the asterisk at the right position
\lstset{basicstyle=\ttfamily,tabsize=4,literate={*}{${}^*{}$}1}
%\lstset{language=C,basicstyle=\ttfamily}
\usepackage{moreverb}
\usepackage{palatino}
\usepackage{multicol}
\usepackage{tabularx}
\usepackage{comment}
\usepackage{verbatim}
\usepackage{color}
\usepackage{enumitem}
\usepackage{tikz}
\usetikzlibrary{arrows,shapes.gates.logic.US,shapes.gates.logic.IEC,calc}

\usepackage[left=3cm, right=3cm, top=4cm, bottom=4cm]{geometry}

\usepackage{graphicx}

%% pdflatex?
\newif\ifpdf
\ifx\pdfoutput\undefined
\pdffalse % we are not running PDFLaTeX
\else
\pdfoutput=1 % we are running PDFLaTeX
\pdftrue
\fi
\ifpdf
\usepackage[pdftex]{graphicx}
\else
\usepackage{graphicx}
\fi
\ifpdf
\DeclareGraphicsExtensions{.pdf, .jpg}
\else
\DeclareGraphicsExtensions{.eps, .jpg}
\fi

\parindent=0cm
\parskip=0cm

\setlength{\columnseprule}{0.4pt}
\addtolength{\columnsep}{2pt}

\addtolength{\textheight}{5.5cm}
\addtolength{\topmargin}{-26mm}
\pagestyle{empty}

%%
%% Sheet setup
%% 
\newcommand{\coursename}{Computer Architecture and Programming Languages}
\newcommand{\courseno}{CO20-320241}
 
\newcommand{\sheettitle}{Homework}
\newcommand{\mytitle}{}
\newcommand{\mytoday}{December 11, 2018}

% Current Assignment number
\newcounter{assignmentno}
\setcounter{assignmentno}{12}

% Current Problem number, should always start at 1
\newcounter{problemno}
\setcounter{problemno}{1}

%%
%% problem and bonus environment
%%
\newcounter{probcalc}
\newcommand{\problem}[2]{
  \pagebreak[2]
  \setcounter{probcalc}{#2}
  ~\\
  {\large \textbf{Problem \arabic{assignmentno}.\arabic{problemno}} \hspace{0.2cm}\textit{#1}} \refstepcounter{problemno}\vspace{2pt}\\}

\newcommand{\bonus}[2]{
  \pagebreak[2]
  \setcounter{probcalc}{#2}
  ~\\
  {\large \textbf{Bonus Problem \arabic{assignmentno}.\arabic{problemno}} \hspace{0.2cm}\textit{#1}} \refstepcounter{problemno}\vspace{2pt}\\}

%% some counters  
\newcommand{\assignment}{\arabic{assignmentno}}

%% solution  
\newcommand{\solution}{\pagebreak[2]{\bf Solution:}\\}

%% Hyperref Setup
\hypersetup{pdftitle={Homework \assignment},
  pdfsubject={\coursename},
  pdfauthor={},
  pdfcreator={},
  pdfkeywords={Computer Architecture and Programming Languages},
  %  pdfpagemode={FullScreen},
  %colorlinks=true,
  %bookmarks=true,
  %hyperindex=true,
  bookmarksopen=false,
  bookmarksnumbered=true,
  breaklinks=true,
  %urlcolor=darkblue
  urlbordercolor={0 0 0.7}
}

\definecolor{mGreen}{rgb}{0,0.6,0}
\definecolor{mGray}{rgb}{0.5,0.5,0.5}
\definecolor{mPurple}{rgb}{0.58,0,0.82}
\definecolor{backgroundColour}{rgb}{0.95,0.95,0.92}

\lstdefinestyle{CStyle}{
    backgroundcolor=\color{backgroundColour},   
    commentstyle=\color{mGreen},
    keywordstyle=\color{magenta},
    numberstyle=\tiny\color{mGray},
    stringstyle=\color{mPurple},
    basicstyle=\footnotesize,
    breakatwhitespace=false,         
    breaklines=true,                 
    captionpos=b,                    
    keepspaces=true,                 
    numbers=left,                    
    numbersep=5pt,                  
    showspaces=false,                
    showstringspaces=false,
    showtabs=false,                  
    tabsize=2,
    language=C
}

\begin{document}
\coursename \hfill Course: \courseno\\
Jacobs University Bremen \hfill \mytoday\\
Dushan Terzikj\hfill
\vspace*{0.3cm}\\
\begin{center}
{\Large \sheettitle{} \assignment}\\
\end{center}

\problem{}{0}
\solution
Classification of programming languages by generation:\\ \\
\begin{tabular}{|c|c|}
     \hline
     Generation&Languages  \\ \hline
     First&  \\ \hline
     Second&  \\ \hline
     Third&C, C++, Java, Basic, Pascal, B  \\ \hline
     Fourth&Ruby, Perl, PHP, Python  \\ \hline
     Fifth&Prolog, Smalltalk  \\ \hline
\end{tabular}\\ \\
Classification of programming languages by type:\\ \\
\begin{tabular}{|c|c|}
     \hline
     Type&Languages  \\ \hline
     Imperative&C, C++, Java, Basic, Pascal, B  \\ \hline
     Declarative&Prolog  \\ \hline
     Von Neumann&C  \\ \hline
     Object-Oriented&Smalltalk, C++, Java, Ruby, PHP, Python  \\ \hline
     Scripting&Perl, PHP, Python, Ruby \\ \hline
\end{tabular}\\ \\

\problem{}{0}
\solution
Assuming that $var$ means any variable necessary to be put in order for the expression to work, we can derive the following grammar (according to the problem statement):\\ \\
\begin{align*}
    trenary&\rightarrow var=condition?expr1:expr2;\\
    condition&\rightarrow true|false|var|relation\\
    relation&\rightarrow var\text{ }rel\text{ }var\\
    rel&\rightarrow <|<=|>|>=|==|!=\\
    expr1&\rightarrow expr\\
    expr2&\rightarrow expr\\
    expr&\rightarrow var|operation\\
    operation&\rightarrow var\text{ }op\text{ }var\\
    op&\rightarrow +|-|*|/
\end{align*}
Where, $T=\{var, +, *, -, /, <, <=, >, >=, ==, !=, =, ?, :, ;, true, false\}$ is the set of terminals, $V=\{trenary, condition, expr1, expr2, expr, relation, operation, rel, op\}$ is the set of variables of the grammar.
\newpage
\problem{}{0}
\solution
\begin{align*}
    whileloop&\rightarrow while(condition)\{statements\}\\
    condition&\rightarrow identifier\text{ }rel\text{ }identifier|identifier\text{ }rel\text{ }constant\\
    rel&\rightarrow <|>|<=|>=|==|!=\\
    statements&\rightarrow statement;\text{ }statements|statement;\\
    statement&\rightarrow var=expr\\
    expr&\rightarrow var|operation\\
    operation&\rightarrow var\text{ }op\text{ }var\\
    op&\rightarrow +|-|*|/\\
\end{align*}
Where, $V=\{whileloop, condition, statements, statement, identifier, constant, rel, expr, operation, op\}$ are variables in the grammar and $T=\{while, (, ), \{, \}, ;, var, <, >, <=, >=, ==, !=, =, +, -, *, /\}$ are terminals of the grammar.

\end{document}