\documentclass[a4paper]{article}
    \usepackage[pdftex]{hyperref}
    \usepackage[latin1]{inputenc}
    \usepackage[english]{babel}
    \usepackage{a4wide}
    \usepackage{amsmath}
    \usepackage{amssymb}
    \usepackage{algorithmic}
    \usepackage{algorithm}
    \usepackage{ifthen}
    \usepackage{listings}
    % move the asterisk at the right position
    \lstset{basicstyle=\ttfamily,tabsize=4,literate={*}{${}^*{}$}1}
    %\lstset{language=C,basicstyle=\ttfamily}
    \usepackage{moreverb}
    \usepackage{palatino}
    \usepackage{multicol}
    \usepackage{tabularx}
    \usepackage{comment}
    \usepackage{verbatim}
    \usepackage{color}
    \usepackage{enumitem}
    \usepackage{tikz}
    \usetikzlibrary{arrows,shapes.gates.logic.US,shapes.gates.logic.IEC,calc}
    
    \usepackage[left=3cm, right=3cm, top=4cm, bottom=4cm]{geometry}
    
    \usepackage{graphicx}
    
    %% pdflatex?
    \newif\ifpdf
    \ifx\pdfoutput\undefined
    \pdffalse % we are not running PDFLaTeX
    \else
    \pdfoutput=1 % we are running PDFLaTeX
    \pdftrue
    \fi
    \ifpdf
    \usepackage[pdftex]{graphicx}
    \else
    \usepackage{graphicx}
    \fi
    \ifpdf
    \DeclareGraphicsExtensions{.pdf, .jpg}
    \else
    \DeclareGraphicsExtensions{.eps, .jpg}
    \fi
    
    \parindent=0cm
    \parskip=0cm
    
    \setlength{\columnseprule}{0.4pt}
    \addtolength{\columnsep}{2pt}
    
    \addtolength{\textheight}{5.5cm}
    \addtolength{\topmargin}{-26mm}
    \pagestyle{empty}
    
    %%
    %% Sheet setup
    %% 
    \newcommand{\coursename}{Computer Architecture and Programming Languages}
    \newcommand{\courseno}{CO20-320241}
     
    \newcommand{\sheettitle}{Homework}
    \newcommand{\mytitle}{}
    \newcommand{\mytoday}{\textcolor{blue}{October 9}, 2018}
    
    % Current Assignment number
    \newcounter{assignmentno}
    \setcounter{assignmentno}{5}
    
    % Current Problem number, should always start at 1
    \newcounter{problemno}
    \setcounter{problemno}{1}
    
    %%
    %% problem and bonus environment
    %%
    \newcounter{probcalc}
    \newcommand{\problem}[2]{
      \pagebreak[2]
      \setcounter{probcalc}{#2}
      ~\\
      {\large \textbf{Problem \textcolor{blue}{\arabic{assignmentno}}.\textcolor{blue}{\arabic{problemno}}} \hspace{0.2cm}\textit{#1}} \refstepcounter{problemno}\vspace{2pt}\\}
    
    \newcommand{\bonus}[2]{
      \pagebreak[2]
      \setcounter{probcalc}{#2}
      ~\\
      {\large \textbf{Bonus Problem \textcolor{blue}{\arabic{assignmentno}}.\textcolor{blue}{\arabic{problemno}}} \hspace{0.2cm}\textit{#1}} \refstepcounter{problemno}\vspace{2pt}\\}
    
    %% some counters  
    \newcommand{\assignment}{\arabic{assignmentno}}
    
    %% solution  
    \newcommand{\solution}{\pagebreak[2]{\bf Solution:}\\}
    
    %% Hyperref Setup
    \hypersetup{pdftitle={Homework \assignment},
      pdfsubject={\coursename},
      pdfauthor={},
      pdfcreator={},
      pdfkeywords={Computer Architecture and Programming Languages},
      %  pdfpagemode={FullScreen},
      %colorlinks=true,
      %bookmarks=true,
      %hyperindex=true,
      bookmarksopen=false,
      bookmarksnumbered=true,
      breaklinks=true,
      %urlcolor=darkblue
      urlbordercolor={0 0 0.7}
    }
    
    \begin{document}
    \coursename \hfill Course: \courseno\\
    Jacobs University Bremen \hfill \mytoday\\
    \textcolor{blue}{Dushan Terzikj}\hfill
    \vspace*{0.3cm}\\
    \begin{center}
    {\Large \sheettitle{} \textcolor{blue}{\assignment}\\}
    \end{center}
    
    \problem{}{0}
    \solution
    \textcolor{blue}{
        \begin{enumerate}[label=(\alph*)]
            \item $14_{10}+36_{10}=001110_2+100100_2=110010_2=50_{10}$
            \item $12_{10}=1100_2$\\
            $-25_{10}=11001_2\xrightarrow{\text{invert bits and add 1}}00111_2$\\
            $01100_2+00111_2=10011_2$\\
            $10011_2\xrightarrow{\text{invert bits and add 1}}01101_2=-13_{10}$
            \item 
            \begin{tabular}{ccccc}
                \hline
                &&0110&1001&  \\
                +&&0101&1000& \\ \hline
                &&1100&0001&$\leftarrow$Invalid code group, add power of 6 \\
                +&&0000&0110& \\ \hline
                &&1100&0111&$\leftarrow$Invalid code group, add power of 6 \\
                +&&0110&0000& \\ \hline
                &0001&0010&0111&$=127_{10}$
            \end{tabular}
            \item
            \begin{tabular}{ccccc}
                \hline
                &0010&0111&0101&  \\
                +&0110&0100&0010& \\ \hline
                &1000&1011&0111&$\leftarrow$Invalid code group, add power of 6 \\
                +&0000&0110&0000& \\ \hline
                &1001&0001&0111&$=917_{10}$
            \end{tabular}
            \item
            \begin{tabular}{|c|c|c|c|}
                \hline
                carry&&1&  \\ \hline
                addend&6&A&F  \\ \hline
                adder&2&3&C  \\ \hline
                sum&8&E&B  \\ \hline
            \end{tabular}
            \item $3AF_{16}\xrightarrow{\text{to bin}}001110101111_2\xrightarrow{\text{invert bits and add 1}}110001010001_2\xrightarrow{\text{to hex}}C51_{16}$ \\ \\
            \begin{tabular}{|c|c|c|c|}
                \hline
                carry&1&&  \\ \hline
                addend&5&9&4  \\ \hline
                adder&C&5&1  \\ \hline
                sum&1&E&5  \\ \hline
            \end{tabular}
        \end{enumerate}
    }
    
    \problem{}{0}
    \solution
    \textcolor{blue}{
        \begin{enumerate}[label=(\alph*)]
            \item add \$t0, \$s0, \$s1 $\leftarrow$ add b and c and store it in a
            \item add \$t0, \$s0, \$s1 $\leftarrow$ add b and c and store it in a \\
            sub \$t0, \$t0, \$s2 $\leftarrow$ subtract a and d and store it in a
            \item add \$t0, \$s0, \$s0 $\leftarrow$ add b and b and store it in a \\
            add \$t0, \$t0, \$s0 $\leftarrow$ add a and b and store it in a
            \item add \$s0, \$s0, 1 $\leftarrow$ add b and 1 and store it in b \\
            add \$t0, \$s0, \$s0 $\leftarrow$ add b and b and store it in a \\
        \end{enumerate}
    }
    \newpage
    \problem{}{0}
    \solution
    \textcolor{blue}{
        \begin{enumerate}[label=(\alph*)]
            \item
            \begin{tabular}{|c|c|c|c|c|c|}
            \hline
                Opcode&Source 1&Source 2&Des&Shamt&Funct  \\ \hline
                000000&10000&10001&01000&00000&100000 \\ \hline
            \end{tabular}
            \item
            \begin{tabular}{|c|c|c|c|c|c|}
            \hline
                Opcode&Source 1&Source 2&Des&Shamt&Funct  \\ \hline
                000000&01000&10010&01000&00000&100010 \\ \hline
            \end{tabular}
        \end{enumerate}
    }
    
    \problem{}{0}
    \solution
    \textcolor{blue}{
        add 20(\$s1), 16(\$s0), 8(\$s0) $$\leftarrow$$ add A[4] and A[2] and store it in B[5] \\ \\
        \textbf{OR} \\ \\
        lw \$t0, 16(\$s0) $\leftarrow$ load A[4] and store it in register \$t0 \\
        lw \$t1, 8(\$s0) $\leftarrow$ load A[2] and store it in register \ \\
        add \$t0, \$t0, \$t1 $\leftarrow$ add the values in registers \$t0 and \$t1 and store it in register \$t0 \\
        sw \$t0, 20(\$s0) $\leftarrow$ store the value in register \$t0 in B[5] \\
    }
    
    \problem{}{0}
    \solution
    \textcolor{blue}{
        add \$t1, \$t0, 2 $\leftarrow$ add 2 to x and store it temporary in \$t0 \\
        add \$t1, \$t1, \$t1 \\
        add \$t1, \$t1, \$t1 $\leftarrow$ get offset 4 \\
        add \$t1, \$t1, \$s0 $\leftarrow$ store the address of A[x+2] in \$t1
        add \$t2, \$t0, 7 \\
        add \$t2, \$t2, \$t2 \\
        add \$t2, \$t2, \$t2 \\
        add \$t2, \$t2, \$s0 $\leftarrow$ same as A[x+2] only this time for A[x+7] and store temporarily in \$t2 \\
        add \$t3, \$t0, \$t0 \\
        add \$t3, \$t3, \$t3 \\
        add \$t3, \$t3, \$s1 $\leftarrow$ store the address of B[x] into \$t3 \\
        lw \$t4, 0(\$t1) $\leftarrow$ load A[x+2] and store it in \$t4 \\
        lw \$t5, 0(\$t2) $\leftarrow$ load A[x+7] and store it in \$t5 \\    
        add \$t6, \$t4, \$t5 $\leftarrow$ add A[x+2] and A[x+7] and store it in register \$t6 \\
        sw \$t6, 0(\$t3) $\leftarrow$ store the value in \$t6 into B[x] \\ 
    }
    
    \problem{}{0}
    \solution
    \textcolor{blue}{
        It would change in a way that the register number of bits will be 4 each, rather than 5. With 4 bits for each register we can represent 16 registers. The 2 bits which are left can be given to the constant part. This makes the constant part to be 18 bits instead of 16, which provides us with the possibility to represent larger constants. 
    }
    
    \end{document}