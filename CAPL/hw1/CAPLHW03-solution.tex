\documentclass[a4paper]{article}
\usepackage[pdftex]{hyperref}
\usepackage[latin1]{inputenc}
\usepackage[english]{babel}
\usepackage{a4wide}
\usepackage{amsmath}
\usepackage{amssymb}
\usepackage{algorithmic}
\usepackage{algorithm}
\usepackage{ifthen}
\usepackage{listings}
% move the asterisk at the right position
\lstset{basicstyle=\ttfamily,tabsize=4,literate={*}{${}^*{}$}1}
%\lstset{language=C,basicstyle=\ttfamily}
\usepackage{moreverb}
\usepackage{palatino}
\usepackage{multicol}
\usepackage{tabularx}
\usepackage{comment}
\usepackage{verbatim}
\usepackage{color}
\usepackage{enumitem}
\usepackage{tikz}
\usetikzlibrary{arrows,shapes.gates.logic.US,shapes.gates.logic.IEC,calc}

\usepackage[left=3cm, right=3cm, top=4cm, bottom=4cm]{geometry}

\usepackage{graphicx}

%% pdflatex?
\newif\ifpdf
\ifx\pdfoutput\undefined
\pdffalse % we are not running PDFLaTeX
\else
\pdfoutput=1 % we are running PDFLaTeX
\pdftrue
\fi
\ifpdf
\usepackage[pdftex]{graphicx}
\else
\usepackage{graphicx}
\fi
\ifpdf
\DeclareGraphicsExtensions{.pdf, .jpg}
\else
\DeclareGraphicsExtensions{.eps, .jpg}
\fi

\parindent=0cm
\parskip=0cm

\setlength{\columnseprule}{0.4pt}
\addtolength{\columnsep}{2pt}

\addtolength{\textheight}{5.5cm}
\addtolength{\topmargin}{-26mm}
\pagestyle{empty}

%%
%% Sheet setup
%% 
\newcommand{\coursename}{Computer Architecture and Programming Languages}
\newcommand{\courseno}{CO20-320241}
 
\newcommand{\sheettitle}{Homework}
\newcommand{\mytitle}{}
\newcommand{\mytoday}{\textcolor{blue}{October 2}, 2018}

% Current Assignment number
\newcounter{assignmentno}
\setcounter{assignmentno}{3}

% Current Problem number, should always start at 1
\newcounter{problemno}
\setcounter{problemno}{1}

%%
%% problem and bonus environment
%%
\newcounter{probcalc}
\newcommand{\problem}[2]{
  \pagebreak[2]
  \setcounter{probcalc}{#2}
  ~\\
  {\large \textbf{Problem \textcolor{blue}{\arabic{assignmentno}}.\textcolor{blue}{\arabic{problemno}}} \hspace{0.2cm}\textit{#1}} \refstepcounter{problemno}\vspace{2pt}\\}

\newcommand{\bonus}[2]{
  \pagebreak[2]
  \setcounter{probcalc}{#2}
  ~\\
  {\large \textbf{Bonus Problem \textcolor{blue}{\arabic{assignmentno}}.\textcolor{blue}{\arabic{problemno}}} \hspace{0.2cm}\textit{#1}} \refstepcounter{problemno}\vspace{2pt}\\}

%% some counters  
\newcommand{\assignment}{\arabic{assignmentno}}

%% solution  
\newcommand{\solution}{\pagebreak[2]{\bf Solution:}\\}

%% Hyperref Setup
\hypersetup{pdftitle={Homework \assignment},
  pdfsubject={\coursename},
  pdfauthor={},
  pdfcreator={},
  pdfkeywords={Computer Architecture and Programming Languages},
  %  pdfpagemode={FullScreen},
  %colorlinks=true,
  %bookmarks=true,
  %hyperindex=true,
  bookmarksopen=false,
  bookmarksnumbered=true,
  breaklinks=true,
  %urlcolor=darkblue
  urlbordercolor={0 0 0.7}
}

\begin{document}
\coursename \hfill Course: \courseno\\
Jacobs University Bremen \hfill \mytoday\\
\textcolor{blue}{Dushan Terzikj}\hfill
\vspace*{0.3cm}\\
\begin{center}
{\Large \sheettitle{} \textcolor{blue}{\assignment}\\}
\end{center}

\problem{}{0}
\solution
\textcolor{blue}{
    \begin{enumerate}[label=(\alph*)]
        \item 
        \begin{align*}
            (M+N)&\cdot (\overline{M}+P)\cdot (\overline{N}+\overline{P})=(M\cdot\overline{M}+M\cdot P+N\cdot\overline{M}+N\cdot P)\cdot (\overline{N}+\overline{P}) \\
            &=(0+M\cdot P\cdot\overline{N}+M\cdot P\cdot\overline{P}+N\cdot\overline{N}\cdot\overline{M}+N\cdot\overline{M}\cdot\overline{P}+N\cdot\overline{N}\cdot P+N\cdot P\cdot\overline{P})\\
            &=M\cdot P\cdot\overline{N}+0+0+N\cdot\overline{M}\cdot\overline{P}+0+0\\
            &=M\cdot P\cdot\overline{N}+N\cdot\overline{M}\cdot\overline{P}
        \end{align*}
        \item
        \begin{align*}
            \overline{A}\cdot B\cdot\overline{C}&+A\cdot B\cdot\overline{C}+B\cdot\overline{C}\cdot D=B\cdot\overline{C}\cdot(\overline{A}+A)+B\cdot\overline{C}\cdot D\\
            &=B\cdot\overline{C}+B\cdot\overline{C}\cdot D\\
            &=B\cdot\overline{C}(1+D)\\
            &=B\cdot\overline{C}
        \end{align*}
        \item
        \begin{align*}
            \overline{(M+N+P)&\cdot Q}=\overline{M\cdot Q+N\cdot Q+P\cdot Q}\\
            &=(\overline{M\cdot Q})\cdot (\overline{N\cdot Q})\cdot (\overline{P\cdot Q})\\
            &=(\overline{M}+\overline{Q})\cdot (\overline{N}+\overline{Q})\cdot (\overline{P}+\overline{Q})
        \end{align*}
        \item
        \begin{align*}
            (\overline{A\cdot B\cdot C+D\cdot E\cdot F})&=(\overline{A\cdot B\cdot C})\cdot (\overline{D\cdot E\cdot F})\\
            &=(\overline{A}+\overline{B}+\overline{C})\cdot (\overline{D}+\overline{E}+\overline{F})
        \end{align*}
        \item
        \begin{align*}
            \overline{(A\cdot\overline{B})+(C\cdot\overline{D})+(E\cdot F)}&=\overline{(A\cdot\overline{B})}\cdot \overline{(C\cdot\overline{D})}\cdot \overline{(E\cdot F)}\\
            &=(\overline{A}+ B)\cdot (\overline{C}+ D)\cdot (\overline{E}+ \overline{F})
        \end{align*}
        \item
        \begin{align*}
             \overline{ \overline{ A + B \cdot \overline{ C}} + D \cdot \overline{ (E + \overline{ F})}} &=
 \overline{ \overline{ A} \cdot (\overline{ B} + C) + D \cdot \overline{ E} \cdot F} \\&=
     \overline{ \overline{ A} \cdot \overline{ B} + \overline{ A} \cdot C + D \cdot \overline{ E} \cdot F} \\&=
     \overline{ \overline{ A} \cdot \overline{ B} \cdot \overline{ \overline{ A} \cdot C} \cdot D \cdot \overline{ E} \cdot F} \\&=
     (A + B) \cdot (A + \overline{ C}) \cdot (\overline{ D} + E + \overline{ F}) 
        \end{align*}
    \end{enumerate}
}

\problem{}{0}
\solution
\textcolor{blue}{
    The logic circuit corresponds to:
    \begin{equation}
        (\overline{ A} \cdot \overline{ B} \cdot D) + (A \cdot \overline{ B} \cdot \overline{ C}) + (\overline{ A} \cdot \overline{ B} \cdot \overline{ C})
    \end{equation}
    Behold, the K-Map for logic circuit (1):
    \begin{center}
        \begin{tabular}{|c|c|c|c|c|}
            \hline
            &$C\cdot D$&$C\cdot\overline{D}$&$\overline{C}\cdot D$&$\overline{C}\cdot\overline{D}$ \\ \hline
            $A\cdot B$&0&0&0&$0_4$ \\ \hline
            $A\cdot\overline{B}$&0&0&1&$1_8$ \\ \hline
            $\overline{A}\cdot B$&0&0&0&$0_{12}$ \\ \hline
            $\overline{A}\cdot\overline{B}$&1&0&1&$1_{16}$ \\ \hline
        \end{tabular}
    \end{center}
    From loop$_{7,8}$ and loop$_{13, 15, 16}$ we get:
    \begin{equation}
        A \cdot \overline{ B} \cdot \overline{ C} + \overline{ A} \cdot \overline{ B} \cdot (D + \overline{ C} \cdot \overline{ D}) 
    \end{equation}
}

\problem{}{0}
\solution
\textcolor{blue}{
    \includegraphics[width=140mm]{3_3.jpeg}
}

\problem{}{0}
\solution
\textcolor{blue}{
    \includegraphics[width=140mm]{3_4.jpeg}
}

\problem{}{0}
\solution
\textcolor{blue}{
    \includegraphics[width=140mm]{3_5.jpeg}
}

\problem{}{0}
\solution
\textcolor{blue}{
    \includegraphics[width=140mm]{3_6.jpeg}
}

\problem{}{0}
\solution
\textcolor{blue}{
    \begin{enumerate}[label=(\alph*)]
        \item Since both latches are PGT, we need the clocks to be positive, i.e. high in order to make a change. For that reason B and C have to be high. Since K is always low in both latches, the only way Y is going to be HIGH is when X is going to be HIGH and that is going to happen if A is HIGH.
        \item CLR keeps a constant negative. However, we can use it in order to reset the values. When the START pulse is negative CLR goes positive, therefore resetting the X and Y value.
        \item
        \includegraphics[width=140mm]{logic.jpg}
    \end{enumerate}
}

\end{document}
