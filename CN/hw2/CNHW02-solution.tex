\documentclass[a4paper]{article}
\usepackage[pdftex]{hyperref}
\usepackage[latin1]{inputenc}
\usepackage[english]{babel}
\usepackage{a4wide}
\usepackage{amsmath}
\usepackage{amssymb}
\usepackage{algorithmic}
\usepackage{algorithm}
\usepackage{ifthen}
\usepackage{listings}
% move the asterisk at the right position
\lstset{basicstyle=\ttfamily,tabsize=4,literate={*}{${}^*{}$}1}
%\lstset{language=C,basicstyle=\ttfamily}
\usepackage{moreverb}
\usepackage{palatino}
\usepackage{multicol}
\usepackage{tabularx}
\usepackage{comment}
\usepackage{verbatim}
\usepackage{color}
\usepackage{enumitem}
\usepackage{tikz}
\usetikzlibrary{arrows,shapes.gates.logic.US,shapes.gates.logic.IEC,calc}

\usepackage[left=3cm, right=3cm, top=4cm, bottom=4cm]{geometry}

\usepackage{graphicx}

%% pdflatex?
\newif\ifpdf
\ifx\pdfoutput\undefined
\pdffalse % we are not running PDFLaTeX
\else
\pdfoutput=1 % we are running PDFLaTeX
\pdftrue
\fi
\ifpdf
\usepackage[pdftex]{graphicx}
\else
\usepackage{graphicx}
\fi
\ifpdf
\DeclareGraphicsExtensions{.pdf, .jpg}
\else
\DeclareGraphicsExtensions{.eps, .jpg}
\fi

\parindent=0cm
\parskip=0cm

\setlength{\columnseprule}{0.4pt}
\addtolength{\columnsep}{2pt}

\addtolength{\textheight}{5.5cm}
\addtolength{\topmargin}{-26mm}
\pagestyle{empty}

%%
%% Sheet setup
%% 
\newcommand{\coursename}{Computer Networks}
\newcommand{\courseno}{CO20-320301}
 
\newcommand{\sheettitle}{Problem Sheet}
\newcommand{\mytitle}{}
\newcommand{\mytoday}{March 8, 2019}

% Current Assignment number
\newcounter{assignmentno}
\setcounter{assignmentno}{2}

% Current Problem number, should always start at 1
\newcounter{problemno}
\setcounter{problemno}{1}

%%
%% problem and bonus environment
%%
\newcounter{probcalc}
\newcommand{\problem}[2]{
  \pagebreak[2]
  \setcounter{probcalc}{#2}
  ~\\
  {\large \textbf{Problem \arabic{assignmentno}.\arabic{problemno}} \hspace{0.2cm}\textit{#1}} \refstepcounter{problemno}\vspace{2pt}\\}

\newcommand{\bonus}[2]{
  \pagebreak[2]
  \setcounter{probcalc}{#2}
  ~\\
  {\large \textbf{Bonus Problem \textcolor{blue}{\arabic{assignmentno}}.\textcolor{blue}{\arabic{problemno}}} \hspace{0.2cm}\textit{#1}} \refstepcounter{problemno}\vspace{2pt}\\}

%% some counters  
\newcommand{\assignment}{\arabic{assignmentno}}

%% solution  
\newcommand{\solution}{\pagebreak[2]{\bf Solution:}\\}

%% Hyperref Setup
\hypersetup{pdftitle={Homework \assignment},
  pdfsubject={\coursename},
  pdfauthor={},
  pdfcreator={},
  pdfkeywords={Computer Networks},
  %  pdfpagemode={FullScreen},
  %colorlinks=true,
  %bookmarks=true,
  %hyperindex=true,
  bookmarksopen=false,
  bookmarksnumbered=true,
  breaklinks=true,
  %urlcolor=darkblue
  urlbordercolor={0 0 0.7}
}

\begin{document}
\coursename \hfill Course: \courseno\\
Jacobs University Bremen \hfill \mytoday\\
Dushan Terzikj\hfill
\vspace*{0.3cm}\\
\begin{center}
{\Large \sheettitle{} \assignment\\}
\end{center}

\problem{}{0}
\solution
For this problem I will be strictly using the lowest ID of a \textbf{bridge} in case there is a tie in the spanning tree. 
\begin{enumerate}[label=(\alph*)]
    \item\begin{enumerate}[label=(\roman*)]
        \item The root of the ST is B1.\\
        Root ports are: P2.2, P8.2, P4.2, P7.1, P3.2, P5.1, P6.1.
        \item The designated ports are all ports which are facing a root ports plus some others. Here is all of the designated ports: P1.1, P1.2, P1.3, P4.3, P3.3, P2.1, P2.3, P5.3, P5.2, P6.3, P2.4, P3.1.
        \item Blocked ports are: P7.2, P6.2, P8.3, P8.1, P4.1.
    \end{enumerate}
    \item\begin{enumerate}[label=(\roman*)]
        \item The root of the ST is B2.\\
        Root ports are: P3.2, P6.1, P8.1, P4.1, P7.1, P5.1.
        \item The designated ports are all ports which are facing a root ports plus some others. Here is all of the designated ports: P2.1, P2.3, P2.4, P6.3, P5.2, P5.3, P3.1, P3.3, P4.3.
        \item Blocked ports are: P8.3, P6.2, P5.3.
    \end{enumerate}
\end{enumerate}

\problem{}{0}
\solution
\begin{enumerate}[label=(\alph*)]
    \item $106280$ packets have been captured, which make $19689056$ bytes.\\ \\
    Looking at the Ethernet broadcast statistics, we have $52837$ packets, which make around $6826$ bytes.\\ \\
    The percentage of broadcast packages is $49.7\%$ and for broadcast bytes $0.035\%$. 
    \item The MAC address sending the bridge PDUs is 00:0c:30:80:d5:55 (Cisco\_80:d5:55). The \textbf{destination} MAC address to which the PDUs are sent is 01:80:c2:00:00:00.\\ \\
    The packets are sent approximately every 2 seconds.\\ \\
    The root bridge identifier is 24576/5/50:57:a8:04:33:40.
    \item Yes, there are other protocols that use LLC encapsulation. They are the following:
    \begin{itemize}
        \item \textbf{ZIP} - Zone Information Protocol
        \item \textbf{IPX RIP} - Internetwork Packet eXchange Routing information protocol
        \item \textbf{IPX SAP} - Internetwork Packet eXchange Service Advertisement Protocol
        \item \textbf{NBIPX} - NetBIOS over Internetwork Packet eXchange
        \item \textbf{BROWSER}
        \item \textbf{DTP} - Dynamic Trunk Protocol
        \item \textbf{CDP} - Cisco Discovery Protocol
    \end{itemize}
\end{enumerate}



\end{document}